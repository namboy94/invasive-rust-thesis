\chapter{Einführung}\label{sec:intro}

Das berühmte Mooresche Gesetz besagt, dass sich die 
``Komplexität integrierter Schaltkreise mit minimalen Komponentenkosten regelmäßig verdoppelt''\cite{mooresLawWikiDe}.
Diese Beobachtung wurde im Jahre 1965 von Gordon Moore formuliert und erwies sich seither größtenteils als korrekt.
Es wird jedoch immer deutlicher, dass dieser Trend sich nicht unendlich fortsetzen kann und eventuell an physische Grenzen stößt.

Eine mögliche Lösung dieses Problems ist es, anstelle von immer schnelleren Prozessoren mehrere Prozessoren parallel zu verwenden.
Dieser Trend ist bereits zu beobachten, nahezu alle herkömmlichen Rechner oder Smartphones verfügen heutzutage über
mehrere Prozessorkerne.

Paralleles Rechnen erhöht jedoch die Komplexität des Rechnersystems als auch der darauf laufenden Software.
%TODO

Invasives Computing bietet eine mögliche  



Rust ist eine relativ neue Programmiersprache, welche verspricht 



Die Parallelisierung von Rechnern liegt immer mehr im Fokus der derzeitigen technologischen Entwicklung.
Dies erfordert die Entwicklung und Nutzung von anderen Programmierparadigmen, welche effektiv
diese neuen Architekturen ausnutzen können. Ein möglicher Ansatz ist hierbei das
invasive Computing. Dieses ermöglicht es dem Programmierer, die Ressourcennutzung
eines Programms feiner zu kontrollieren.

Das IRTSS Betriebssystem, welches eine beispielhafte Implementierung für ein solches
invasives System bietet, unterstützt derzeit die Verwendung der Programmiersprachen
C, C++ als auch X10. Hiermit wird die Einführung der Programmiersprache Rust als
weitere unterstützte Sprache vorgeschlagen. Diese Sprache hat einige wünschenswerte Merkmale, 
welche interessant für den Gebrauch im invasiven Computing sind.

Verwandte Arbeit ausm Wiki