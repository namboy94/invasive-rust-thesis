\chapter{Einführung}\label{sec:intro}

Die Parallelisierung von Rechnern liegt immer mehr im Fokus der derzeitigen technologischen Entwicklung.
Dies erfordert die Entwicklung und Nutzung von anderen Programmierparadigmen, welche effektiv
diese neuen Architekturen ausnutzen können. Ein möglicher Ansatz ist hierbei das
invasive Computing. Dieses ermöglicht es dem Programmierer, die Ressourcennutzung
eines Programms feiner zu kontrollieren.

Das IRTSS Betriebssystem, welches eine beispielhafte Implementierung für ein solches
invasives System bietet, unterstützt derzeit die Verwendung der Programmiersprachen
C, C++ als auch X10. Hiermit wird die Einführung der Programmiersprache Rust als
weitere unterstützte Sprache vorgeschlagen. Diese Sprache hat einige wünschenswerte Merkmale, 
welche interessant für den Gebrauch im invasiven Computing sind.
