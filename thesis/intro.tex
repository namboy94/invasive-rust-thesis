\chapter{Einführung}\label{sec:intro}

Gordon Moore hat im Jahre 1965 das Mooresche Gesetz formuliert, welches besagt, dass sich die Integrationsdichte
von integrierten Schaltkreisen in regelmäßigen Intervallen verdoppelt\cite{mooresLawPastPresentFuture}.
Diese Beobachtung erwies sich bisher als korrekt, allerdings wird es nach und nach wahrscheinlicher,
dass die heutigen Fertigungsmethoden diesen Trend nicht unendlich fortführen können, da diese an
physische Grenzen stoßen werden.\cite{endOfMooresLaw}

Um zukünftig trotzdem eine verbesserte Rechenleistung zu erzielen, wird unter anderem auf Parallelrechner gesetzt.
Der Grundgedanke dahinter ist es, mehrere Recheneinheiten zu verwenden, welche gemeinsam eine Aufgabe
lösen. Somit wird keine Verbesserung der einzelnen Recheneinheiten benötigt, um eine verbesserte Rechenleistung
zu erzielen.

Der Trend zum parallelen Rechnen kann bereits heute beobachtet werden,
so sind moderne PCs oder Smartphones generell alle mit Mehrkernsystemen ausgestattet.
Bei Grafikprozessoren kommen mittlerweile bereits Tausende einzelner Rechenkerne zum Einsatz,
beispielsweise verfügt die Nvidia GTX 1080 GPU laut Spezifikation \cite{nvidia1080specs} über 2560 Kerne.

Durch den Einsatz von parallelen Rechnern entstehen jedoch auch Kosten für den Programmierer,
denn das Rechnersystem wird hierdurch komplexer.
Während der Entwicklung muss sichergestellt werden, dass durch die gleichzeitige Verarbeitung durch die einzelnen
Recheneinheiten keine Fehler entstehen. Die individuellen Prozessorkerne müssen also miteinander
kommunizieren und die Hardwareressourcennutzung untereinander koordinieren.

Um die zusätzliche Komplexität des parallelen Rechnens für den Programmierer zu verringern,
müssen neue Techniken oder Programmierparadigmen entwickelt werden.
Eine solche wäre das invasive Rechnen, welches es einem Programmierer erlaubt,
die Ressourcen in parallelen Systemen besser zu nutzen.
Vor allem bei Systemen mit vielen Kernen ist dieses Paradigma ein interessanter Lösungsansatz.
So kann man beispielsweise auf einer Nvidia GTX 1080 GPU mit insgesamt 2560 Kernen ein Programm
ausführen, welches ein Problem bearbeitet, das auf genau 1000 dieser Kerne ausgeführt wird.
Gleichzeitig kann ein anderes Problem auf einer weiteren Untermenge der Kerne ausgeführt werden.

Die derzeit existierenden invasiven Systeme unterstützen nur die Programmiersprachen C, C++ und X10.
Eine interessante Addition hierzu wäre die Sprache Rust,
welche einen Fokus auf die Sicherheit vor schwerwiegenden Programmierfehlern, die vor 
allem durch Zugriffe auf ungültige Speicherregionen ausgelöst werden, legt.

Damit Rust im Kontext des invasiven Rechnens verwendet werden kann,
werden zunächst die Grundlagen der Programmiersprache Rust,
des invasiven Rechnens und der Prozessorarchitektur SPARC-V8 erarbeitet.
Anschließend werden Werkzeuge zum Erstellen von invasiven
Rust Programmen entwickelt, welche das Ziel verfolgen,
Rust effektiv in Verbindung mit diesem Programmierparadigma zu verwenden.
Abschließend muss der tatsächliche Wert dieses Unterfangens evaluiert und bewertet werden.
