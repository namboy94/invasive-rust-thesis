\chapter{Einführung}\label{sec:intro}

Das berühmte Mooresche Gesetz besagt, dass sich die 
``Komplexität integrierter Schaltkreise mit minimalen Komponentenkosten regelmäßig verdoppelt''\cite{mooresLawWikiDe}.
Diese Beobachtung wurde im Jahre 1965 von Gordon Moore formuliert und erwies sich seither größtenteils als korrekt.
Dieser Trend wird sich mit den derzeit verwendeten Fertigungsmethoden jedoch nicht unendlich fortsetzen können und eventuell an
physische Grenzen stoßen. Um zukünftig trotzdem eine verbesserte Rechenleistung zu erzielen, wird unter anderem auf
Parallelrechner gesetzt. Der Grundgedanke dahinter ist es, mehrere Recheneinheiten zu verwenden, welche gemeinsam ein Problem
abarbeiten und somit nicht eine Verbesserung der einzelnen Recheneinheiten benötigen, um eine verbesserte Leistung aufzuweisen.

Der Trend zum parallelen Rechnen kann bereits seit einiger Zeit beobachtet werden; so sind moderne PCs oder Smartphones
generell alle mit Mehrkernsystemen ausgestattet. Bei Grafikprozessoren kommen mittlerweile bereits Tausende einzelne Kerne
zum Einsatz, so hat beispielsweise die Nvidia GTX 1080 GPU laut Spezifikation \cite{nvidia1080specs} 2560 Kerne verbaut.

Durch den Einsatz von parallelem Rechnen entstehen jedoch auch Kosten für den Programmierer, denn das Rechnersystem wird
hierdurch komplexer. Der Programmierer muss sicherstellen dass durch die gleichzeitige Verarbeitung durch die einzelnen
Recheneinheiten keine Fehler entstehen, es muss also die Kommunikation zwischen den Prozessorelementen bestehen, um solche
Fehler zu vermeiden. Außerdem muss der Programmierer auch die vorliegenden Hardwareressourcen

Um die zusätzliche Komplexität des parallelen Rechnens für den Programmierer zu verringern, müssen neue Techniken oder
Programmierparadigmen entwickelt werden. Eine solche Idee ist das Invasive Computing, welches es einem Programmierer
erlaubt, die Ressourcen in Parallelen Systemen besser zu nutzen. Vor allem bei Systemen mit vielen Kernen ist dieses Paradigma
eine interessanter Lösungsansatz. So kann man beispielsweise auf einer Nvidia GTX 1080 GPU mit insgesamt 2560 Kernen ein Programm
ausführen, welches dann ein Problem bearbeitet welches auf genau 1000 dieser Kerne ausgeführt wird.

Das Invasive Computing unterstützt derzeit nur die Programmiersprachen C, C++ und X10. Eine interessante Addition hierzu wäre
die Sprache Rust, welche einen Fokus auf die Vermeidung von Fehlern legt.

\section{Verwandte Arbeiten}

Eine verwandte Arbeit ist ``Invasive Computing—An Overview'' von Jürgen Teich, Jörg Henkel, Andreas Herkersdorf, Doris Schmitt-Landsiedel, Wolfgang Schröder-Preikschat und Gregor Snelting. Diese illustriert die Grundkonzepte hinter dem Invasiven Computing.