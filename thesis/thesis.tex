\documentclass[parskip=full,12pt,a4paper,twoside,headings=openright]{scrreprt}
% switch to scrbook if you want roman page numbers for the front matter
% however scrbook has no 'abstract' environment!
% if your thesis is in english, use "parskip=no" instead

% binding correction (BCOR) von 1cm für Leimbindung
\KOMAoptions{BCOR=1cm}
\KOMAoptions{draft=yes}

\usepackage[utf8]{inputenc} % encoding of sources
\usepackage[T1]{fontenc}
\usepackage{studarbeit}
\usepackage{xcolor}
\usepackage{listings}
\usepackage[section]{placeins}

\title{Invasives Rust}
\author{Hermann Heinz Erich Krumrey}
\thesistype{Bachelorarbeit}
\zweitgutachter{Prof.~Dr.-Ing.~Jörg~Henkel}% for compiler stuff
\betreuer{Dipl.-Inform.~Andreas~Zwinkau}
\coverimage{coverimage/cover.png}

\newcommand{\libFIRM}{lib\textsc{Firm}}

\begin{document}

\begin{otherlanguage}{ngerman} % Titelseite ist immer auf Deutsch
\mytitlepage
\end{otherlanguage}

\begin{abstract}
\begin{center}\Huge\textbf{\textsf{Zusammenfassung}}
\end{center}
\vfill

Die Parallelisierung von Rechnern liegt immer mehr im Fokus der derzeitigen technologischen Entwicklung.
Es erfordert die Entwicklung und Nutzung von neuen Programmierparadigmen, welche effektiv
diese neuen Architekturen ausnutzen können. Ein möglicher Ansatz ist hierbei das
invasive Computing. Dieses ermöglicht es dem Programmierer, die Ressourcennutzung
eines Programms explizit und dynamisch zu kontrollieren.

Das \textit{OctoPOS} Betriebssystem und die invasive Middleware \textit{iRTSS},
welche eine beispielhafte Implementierung für ein solches
invasives System bietet, unterstützen derzeit die Verwendung der Programmiersprachen
C, C++ als auch X10. Hiermit wird die Einführung der Programmiersprache Rust als
weitere unterstützte Sprache vorgeschlagen.

Im Rahmen dieser Arbeit wurde das \textit{octorust} Programm und die zugehörige \textit{octolib} Bibliothek
entwickelt, welche es ermöglichen, Rust auf invasiven Systemen zu verwenden. Der Einsatz von Rust
ermöglicht das speichersichere Programmieren ohne einen \textit{Garbage Collector}, wobei in Folge dessen
keine Kompromisse bezüglich der Laufzeit eingegangen werden müssen. Daher bietet Rust einen
Vorteil gegenüber den bereits von \textit{OctoPOS} und \textit{iRTSS} unterstützten Sprachen.

Vorläufige Messungen haben ergeben, dass Rust in Situationen, in denen häufige Speicherallokationen und
-deallokationen vorkommen, eine über dreißig mal kürzere Laufzeit als X10 aufweisen kann, wobei bei mathematischen
Berechnungen keine eindeutige Leistungseinstufung ersichtlich ist.

\vfill

\tiny
Das Titelbild dieser Arbeit ist eine modifizierte Version des offiziellen Logos der Rust Programmiersprache. \\
Dieses ist unter der \textit{Creative Commons Attribution License (CC-BY)}
(\url{https://creativecommons.org/licenses/by/4.0/}) \\
zur Benutzung und Modifikation freigegeben.

\end{abstract}

\tableofcontents

\chapter{Einführung}\label{sec:intro}

Einführung
%TODO Einführung


\chapter{Grundlagen und Verwandte Arbeiten}\label{sec:basics}

\section{Rust}

Rust ist eine Programmiersprache, welche von Mozilla Research entwickelt wird \cite(rustWikDe).
Die Entwicklung der Sprache begann als persönliches Projekt des Mozilla-Mitarbeiters Graydon Hoare und wird
seit 2009 von Mozilla unterstützt \cite(rustWikDe). 2010 erschien die erste öffentliche Version der Sprache,
im Jahr 2015 wurde die erste stabile Version veröffentlicht \cite(rustWikDe). 
Sie erfreut sich in der jüngsten Vergangenheit wachsende Beliebtheit bei Programmierern aller Art.
%TODO Statistik dass Rust beliebt ist


\subsection{Motivation}

Eine der Kernziele der Rust Programmiersprache ist es, sichere Speicherzugriffe zu gewährleisten. Dies wird mithilfe des sogenannten
``Ownership''-modells erreicht. Dieses Modell ermöglicht es, fehlerhafte Speicherzugriffe und Pufferüberläufe zu vermeiden, ohne
dabei von einem Garbage Collector Gebrauch zu machen. Dies macht Rust zu einer interessanten Alternative zu Low-Level Sprachen
wie C, welche nicht sicher bezüglich der Speicherverwaltung sind, als auch zu Sprachen mit Garbage Collector wie X10, welche durch
den Garbage Collector in gewissen Situationen, beispielsweise bei Echtzeitsystemen, nicht ideal sind.

Ein weiteres Ziel der Sprache ist die Geschwindigkeit. Rust soll vergleichbare Geschwindigkeiten zu anderen Low-Level Sprachen wie
C erreichen und obendrein effizientes Einbinden mit anderen Sprachen ermöglichen.

Nebenläuigkeit ist ein weiteres Hauptziel der Sprache.

Rust bietet Abstraktionen, welche im Gegensatz zu Sprachen wie Python, nicht oder nur gering die Geschwindigkeit des Programm beinflussen.

\subsection{Grundlegende Eigenschaften der Sprache}

Die Programmiersprache Rust lehnt sich syntaktisch an andere C-artige Sprachen an, unterscheidet sich aber hierbei in einigen
Aspekten. So sind in `if`/`else` Blöcken beispielsweise die runden Klammern beispielsweise Optional, so wie es auch in
Sprachen wie Python der Fall ist.

Typsystem

Mehrere Pardigmen (OOP, Funktional)

\subsection{Eigentums/Ownership Modell}

\subsection{Architektur/Compiler}

Der offizielle Rust Compiler rustc

Zum Abhängigkeitsmanagement und Distribution wurde das Tool `cargo` entwickelt. Es ermóglicht es 

Etwas zum Cross Compiling



\section{SPARC}

Sparc

\section{Invasives Computing}

Invasives Computing ist ein paralleles Programmiermodell, welches es ermöglicht, temporär Ressourcen auf einem Parallelrechner
zu beanspruchen und anschließend wieder freizugeben.Hierbei gibt es drei wesentliche Phasen, die `invade`, `infect` und `retreat` Phasen. 

In der `invade` Phase werden zunächst Ressourcen für das laufende Programm reserviert. Welche Ressourcen genau reserviert werden,
wird durch `constraints` bestimmt.

Anschließend wird in der `infect` Phase die Funktion auf den reservierten Prozessorelementen ausgeführt.

Werden die reservierten Ressourcen nicht mehr benötigt, so kann man diese in der `retreat` Phase wieder freigeben.

\chapter{Entwurf und Implementierung}\label{sec:impl}

Im folgenden wird die Implementierung des octorust Compilers und der octolib Bibliothek erläutert, welche
den Gebrauch von Rust im invasiven Computing ermöglichen.

\section{Rust auf der SPARCv8 Platform}

Rust wird derzeit nicht offiziell auf der SPARCv8 Platform unterstützt, da Rust jedoch als Backend LLVM verwendet und dieses
die Platform unterstützt, ist es auch möglich Rust auf dieser Architektur auszuführen.

Rust erlaubt es, Programme ohne die Standardbibliothek zu implementieren, welches in diesem Fall notwendig ist, da die
Standardbibliothek nicht trivial auf jeder beliebigen Platform verwendbar ist. 

nostd core

rustc json

cargo json

\section{Erstellung des octorust Compilers}

Um das komplizierte Vorgehen bei der ``nostd'' Kompilierung und das anschließende Verlinken mit der IRTSS Runtime zu
vereinfachen, wurde ein Python Programm namens ``octorust'' implementiert. Dieses erlaubt es,

struktur

linken

\section{Bindings zur C API}

Im folgenden wurde eine Rust Bibliothek geschrieben, welche

\section{Rust-spezifische Verbesserungen}

AgentClaim / Constraints Structs

infect

Implizites retreat

Closures


\chapter{Evaluation}\label{sec:eval}

\section{Vergleich mit C}

\section{Vergleich mit X10}
\chapter{Fazit und Ausblick}\label{sec:conclusion}

Im folgenden werden die Ergebnisse der Evaluation bewertet und ein Ausblick auf die Zukunft der Programmiersprache Rust
im Bezug zum invasiven Computing geboten.

\section{Fazit}

Betrachtet man die Ergebnisse der Evaluation, so erkennt man, dass Rust in gewissen Aspekten C oder X10 vorzuziehen ist.

Zum einen eliminiert Rust einige Fehlerquellen, welche in C zu undefiniertem Verhalten führen können. In Rust muss der
Programmierer nicht selbst auf solche Fehler achten und wird teils bereits vom Compiler auf Fehler hingewiesen.
Dies erlaubt es, sicherere und weniger fehleranfällige Programme zu schreiben.

Des Weiteren weist Rust meist ein besseres Laufzeitverhalten auf als X10 es tut, vor allem in Situationen
in denen der Garbage-Collector notwendig wird. Ein Garbage Collector ist zudem für Situationen, in denen Pausen nicht
erwünscht sind, beispielsweise bei Echtzeitsystemen, ein Problem, denn diese verletzt die Verfügbarkeitsbedingung des Systems.
Außerdem weist das Kompilieren mit octorust weitaus kürzere Kompilierungsdauern als der x10i-Compiler auf, welches bei
der Entwicklung von Programmen unter Umständen ein nicht insignifikantes Zeitersparnis mit sich bringen kann. Zum Gebrauch auf
Geräten mit sehr limitiertem Speicherplatz wäre Rust ebenfalls X10 vorzuziehen, da kompilierte Rust-Programme weniger Speicherplatz
in Anspruch nehmen. Allerdings führt C in dieser Hinsicht weiterhin.

Negativ zu betrachten ist die starke Abhängigkeit von der C-Schnittstelle, welche die Sicherheit von Rust teils aufgibt.
Die Funktionen der C-Schnittstelle bieten nämlich keine Garantien über die Sicherheit und die häufige Nutzung von
Void-Zeigern zum Übertragen von Daten ist ebenfalls ein Problem, welches zu Fehlern führen kann.

\section{Ausblick}

Da Rust noch eine relativ junge Programmiersprache ist, existieren noch viele Bibliotheken für diese nicht. Zwar können mithilfe
der Foreign Function Interface Bibliotheken, die in anderen Sprachen geschrieben worden sind, eingebunden werden, diese bieten jedoch
nicht die Sicherheitsgarantien wie Rust es tut. In dieser Hinsicht sollte sich die Lage jedoch in der Zukunft verbessern, wenn Rust
weiterhin gerne von Entwicklern genutzt wird und eventuell großflächiger zum Einsatz kommt.

Eine weitere Möglichkeit in der Zukunft ist die Portierung der Standarbibliothek auf die SPARC-V8 Architektur. Die Standardbibliothek
bietet einige hilfreiche Konstrukte und Funktionen, die das Programmieren erleichtern und sicherer machen. Möglich wäre es, dass die 
Entwickler der Rust Programmiersprache selbst die SPARC-V8 Architektur in der Zukunft unterstützen und so die
Standarbibliothek portiert wird, alternativ wäre es möglich dass die Standarbibliothek im Rahmen des invasiven Computing
portiert wird, sollte genug Interesse daran bestehen.

Um die starke Abhängigkeit von IRTSS' C-Schittstelle zu verringern, können weitere Abstraktionen erstellt werden, welche von den
besonderen Eigenschaften der Rust-Programmiersprache Gebrauch machen. Dass dies möglich ist, hat die Implementierung des X10-Compilers
x10i bewiesen, denn dieser spricht die exakt selbe C-Schnittstelle an, bietet jedoch robuste Abstraktionen über diese und erlaubt so
das Entwickeln von X10-Programmen ohne dass der Programmierer sich mit dieser befassen muss.


\bibliographystyle{ieeetr}
\bibliography{bib}

\begin{otherlanguage}{ngerman}
\chapter*{Erklärung}
\pagestyle{empty}

  \vspace{20mm}
  Hiermit erkläre ich, \theauthor, dass ich die vorliegende Bachelorarbeit selbst\-ständig
verfasst habe und keine anderen als die angegebenen Quellen und Hilfsmittel
benutzt habe, die wörtlich oder inhaltlich übernommenen Stellen als solche kenntlich gemacht und
die Satzung des KIT zur Sicherung guter wissenschaftlicher Praxis beachtet habe.
  \vspace{20mm}
  \begin{tabbing}
  \rule{4cm}{.4pt}\hspace{1cm} \= \rule{7cm}{.4pt} \\
 Ort, Datum \> Unterschrift
  \end{tabbing}
\end{otherlanguage}

\chapter*{Danksagung}
\pagestyle{empty}

Ich danke meinen Eltern Janine und Heinz als auch meinem Bruder Michael, die mich meine gesamtes Leben lang durch dick und dünn
begleitet und unterstützt haben. Außerdem danke ich meinen guten Freunden Simon Eherler und Frederick Horn, ohne die ich nicht der
Mensch wäre der ich heute bin. Ich danke meinen Kommilitonen Marius Take, Johannes Bucher, Thomas Schmidt und Daniel Mockenhaupt,
ohne die mein Studium am KIT nicht halb so schön wäre. Ich danke meiner "`Ersatzfamilie"', der Familie Eherler,
die immer einen Platz in ihrer Mitte für mich hat. Ich danke meinem Betreuer Andreas Zwinkau, welcher
mich freundlich und hilfreich durch die Erstellung dieser Arbeit begleitet hat. Und zu guter Letzt
danke ich dem Karlsruher Institut für Technologie, welches es mir erst ermöglichte, dieses Studium zu absolvieren.

\pagestyle{fancy}
\appendix

%\input{appendix.tex}

\end{document}
