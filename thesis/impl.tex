\chapter{Entwurf und Implementierung}\label{sec:impl}

Im folgenden wird die Implementierung des octorust Compilers und der octolib Bibliothek erläutert, welche
den Gebrauch von Rust im invasiven Computing ermöglichen.

\section{Rust auf der SPARCv8 Platform}

Rust wird derzeit nicht offiziell auf der SPARCv8 Platform unterstützt, da Rust jedoch als Backend LLVM verwendet und dieses
die Platform unterstützt, ist es auch möglich Rust auf dieser Architektur auszuführen.

Rust erlaubt es, Programme ohne die Standardbibliothek zu implementieren, welches in diesem Fall notwendig ist, da die
Standardbibliothek nicht trivial auf jeder beliebigen Platform verwendbar ist. 

nostd core

rustc json

cargo json

\section{Erstellung des octorust Compilers}

Um das komplizierte Vorgehen bei der ``nostd'' Kompilierung und das anschließende Verlinken mit der IRTSS Runtime zu
vereinfachen, wurde ein Python Programm namens ``octorust'' implementiert. Dieses erlaubt es,

struktur

linken

\section{Bindings zur C API}

Im folgenden wurde eine Rust Bibliothek geschrieben, welche

\section{Rust-spezifische Verbesserungen}

AgentClaim / Constraints Structs

infect

Implizites retreat

Closures

