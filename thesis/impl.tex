\chapter{Entwurf und Implementierung}\label{sec:impl}

Im folgenden werden Programme und Bibliotheken entwickelt, welche es ermöglichen, Rust auf der von
\textit{OctoPOS} und \textit{iRTSS} zur Verfügung gestellten invasiven Platform verwenden zu können.
Dies ermöglicht es Programme, welche vom invasiven Computing Gebrauch machen, in der
Programmiersprache Rust zu schreiben.

\section{Rust auf der SPARC LEON Architektur}\label{sec:rust_on_sparc}

Rust wird derzeit nicht offiziell auf der SPARC-V8 Architektur unterstützt. Rust verwendet jedoch als Backend LLVM,
welches diese Architektur unterstützt und daher ist es prinzipiell möglich,
Rust-Programme für diese Architektur zu kompilieren.
Die Rust-Standardbibliothek ist allerdings nicht trivial auf andere Architekturen zu portieren.
Um diese Hürde zu umgehen bietet Rust die Funktion,
Programme mit einer minimalen, platformunabhängigen Untermenge der Standardbibliothek zu kompilieren.
Diese Funktion wird hier ausgenutzt,
um eine minimale Implementierung der Programmiersprache auf die SPARC-V8 Architektur zu portieren.

Um ein Rust Programm für eine nicht offiziell unterstütze Architektur zu kompilieren,
muss man zuerst die \textit{libcore} Bibliothek
für die Zielarchitektur kompilieren. Um dies zu bewerkstelligen, benötigt man eine JSON-Datei, welche
dem Compiler die nötigen Informationen bezüglich der Zielarchitektur zur Verfügung stellt.
Eine solche JSON Datei für die
SPARC-V8 Architektur wird beispielsweise in Listing \ref{code:leon_json}  sieht beispielsweise wie folgt aus\cite{initialSparcSupportGithub}:

\begin{lstlisting}[float,caption={
Eine beispielhafte JSON-Spezifikationsdatei für die SPARC-V8 Architektur.
\cite{initialSparcSupportGithub}
},label=code:leon_json]
{
    "arch": "sparc",
    "data-layout": "E-m:e-p:32:32-i64:64-f128:64-n32-S64",
    "executables": true,
    "llvm-target": "sparc",
    "os": "none",
    "panic-strategy": "abort",
    "target-endian": "big",
    "target-pointer-width": "32",
    "linker-flavor": "ld",
    "linker": "Pfad zum SPARC-V8 C-Linker",
    "link-args": [
        "-nostartfiles"
    ]
}
\end{lstlisting}

Außerdem wird ein C-Linker, beispielsweise \texttt{\textsc{\textbf{gcc}}},
für die SPARC-V8 Architektur benötigt. Der Pfad zu diesem Linker
muss in der \texttt{\textsc{\textbf{linker}}}-Option der JSON-Datei angegeben werden.

Sobald man diese JSON Datei erstellt hat, kann man die \textit{libcore} Bibliothek mit dem
\texttt{\textsc{\textbf{cargo build}}} Befehl kompilieren.
Um dies für SPARC-V8 zu tun, verwendet man als Parameter für die \texttt{\textsc{\textbf{-{}-target}}}-Option
den Pfad zur Spezifikations-JSON-Datei.
Derzeit ist es nicht möglich, \textit{libcore} mit dem stabilen Rust-Compiler zu kompilieren,
da diese Bibliothek Funktionalitäten verwendet, welche auf dem stabilen Compiler deaktiviert sind.
Daher muss ein Nightly-Compiler installiert werden, welches
dank \texttt{\textsc{\textbf{rustup}}} nutzerfreundlich gestaltet ist.
Außerdem sollte die Version der \textit{libcore} Bibliothek
mit der Version des nightly-Compilers übereinstimmen,
um versionsabhängige Konflikte bei der Kompilierung zu vermeiden.

Nachdem die \textit{libcore} Bibliothek erfolgreich kompiliert wurde, kann man,
wenn \texttt{\textsc{\textbf{rustup}}} verwendet wurde um \texttt{\textsc{\textbf{rustc}}} und
\texttt{\textsc{\textbf{cargo}}} zu installieren, 
die resultierende \texttt{\textsc{\textbf{.rlib}}}-Datei anderen Rust-Programmen
beim Kompilieren zur Verfügung stellen,
indem man diese in das korrekte Unterverzeichnis im lokalen \texttt{\textsc{\textbf{.rustup}}} Verzeichnis kopiert.
Alternativ kann man das \textit{libcore} Cargo-Projekt direkt durch die Abhängigkeiten in der Cargo.toml Datei
eines Cargo Projekts einbinden.

Soll ein Rust-Programm \textit{libcore} anstelle der Standardbibliothek verwenden, muss man dies
mit der Anweisung \texttt{\textsc{\textbf{\#![no\char`_std]}}}
am Anfang des Programms kenntlich machen.
Außerdem müssen die Funktionen \texttt{\textsc{\textbf{eh\char`_personality}}},
\texttt{\textsc{\textbf{eh\char`_unwind\char`_resume}}} und
\texttt{\textsc{\textbf{panic\char`_fmt}}} in diesem Programm manuell implementiert werden.
Nennenswert ist vor allem letztere, denn diese Funktion
wird aufgerufen, sobald ein Programm nach einer unbehandelten Ausnahme kontrolliert die Ausführung beendet.
Der Anfang eines zu kompilierenden Programms muss also aussehen wie das Beispiel in Listing \ref{code:minimal_nostd}.

\begin{lstlisting}[float,caption={
Der Beginn eines Rust-Programms, welches nicht die Standardbibliothek verwendet.
},label=code:minimal_nostd]
#![feature(lang_items, libc)]
#![no_std]
#![no_main]

#[lang = "eh_personality"] extern fn eh_personality() {}
#[lang = "eh_unwind_resume"] extern fn eh_unwind_resume() {}
#[lang = "panic_fmt"] fn panic_fmt() -> ! { loop {} }
\end{lstlisting}

Um dann das Programm zu kompilieren, verwendet man entweder \texttt{\textsc{\textbf{rustc}}} bei eigenständigen
\texttt{\textsc{\textbf{.rs}}} Dateien oder den \texttt{\textsc{\textbf{cargo build}}} Befehl für Cargo Projekte. Beiden Befehlen muss wie auch beim Kompilieren von \textit{libcore} die 
Spezifikations-JSON-Datei als Argument für die \texttt{\textsc{\textbf{-{}-target}}}-Option übergeben werden. 

\section{Erstellung des \textit{octorust} Hilfsprogramms}

Um das relativ komplizierte und fehleranfällige Kompilieren für Rust-Programme
auf der SPARC-V8 Architektur zu vereinfachen, als
auch besagte Rust-Programme zusammen mit dem \textit{OctoPOS} Betriebssytem und dem \textit{iRTSS} zu verwenden,
wurde ein Python-Programm namens \textit{octorust} geschrieben, welche diese Schritte
vereinfacht.
Das Programm wurde in Python geschrieben, da Pythons Standardbibliothek viele nützliche Funktionen zur Manipulation
von Dateien bietet, welches sich für diesen Zweck als hilfreich erwiesen hat. Außerdem bietet Python mit 
\texttt{\textsc{\textbf{argparse}}} ein praktisches Modul zum Verarbeiten von Kommandozeilen-Argumenten.
Dieses ermöglicht es, eine Vielzahl an Kommandozeilenargumenten mit minimalem Programmieraufwand anzubieten.

Das Programm verwendet \texttt{\textsc{\textbf{python-setuptools}}} und eine dafür konfigurierte 
\texttt{\textsc{\textbf{setup.py}}} Datei, um das Programm lokal zu installieren.
Während des Installationsprozesses wird zum einen das \textit{octorust}-Programm selbst lokal als Python Modul 
installiert und das zugehörige \texttt{\textsc{\textbf{octorust}}} Skript als ausführbare Datei dem Nutzer zur 
Verfügung gestellt.
Gleichzeitig wird ein Verzeichnis namens \texttt{\textsc{\textbf{.octorust}}}
im Heimverzeichnis des derzeitigen Nutzers erstellt.
Dieses fungiert als Speicherort für \textit{OctoPOS/iRTSS}-builds, die in Kapitel \ref{sec:octolib} genauer
beschriebene \textit{octolib} Bibliothek und im Falle dass kein solcher zuvor installiert war, einen SPARC-V8
\texttt{\textsc{\textbf{gcc}}}.
Zudem werden für alle unterstützten Architekturen die \textit{libcore} und \textit{libc} Bibliotheken
kompiliert und in den Installationspfad des derzeit verwendeten \texttt{\textsc{\textbf{rustup}}}-Toolchains kopiert.

\textit{Octorust} bietet die folgenden Optionen:
\begin{description}

	\item[-h, -{}-help]
	Diese Option druckt einen Nutzungshinweis inklusive aller möglichen Kommandozeilenoptionen.
	
	\item[-a. -{}-architecture]
	Diese Option ermöglicht es, eine IRTSS Zielarchitektur zu wählen. Zur Wahl stehen \textit{x86guest},
	\textit{x64native} als auch \textit{leon}.
	Sollte diese Option nicht angegeben werden, wird standardmäßig für die \textit{x86guest} Architektur kompiliert.
	
	\item[-v, -{}-variant]
	Diese Option ermöglicht es, eine \textit{OctoPOS/iRTSS}-Variante zu wählen,
	beispielsweise \textit{generic} oder \textit{4t5c-nores-chipit-w-iotile}.
	Sollte diese Option nicht angegeben werden, wird je nach gewählter Architektur
	eine passende Standardvariante verwendet.
	Im Falle von \textit{x86guest} und \textit{x64native} wird die Variante \textit{generic} gewählt und für 
	\textit{leon} die Variante \textit{4t5c-nores-chipit-w-iotile}.
	
	\item[-o, -{}-output]
	Mit dieser Option kann der Pfad zur resultierenden Ausgabedatei explizit angegeben	werden.
	Ansonsten ermittelt \textit{octorust} automatisch einen sinnvollen Ausgabenamen,
	beispielsweise \texttt{\textsc{\textbf{foo.out}}} für ein Cargo-Project mit dem Namen
	\textit{foo}.
	
	\item[-k, -{}-keep]
	Wird diese Option verwendet, werden jegliche temporäre Dateien,
	die während dem Kompilieren und dem Linken erstellt werden im
	Anschluss nicht gelöscht.
	Dies beinhaltet unter anderem erstellte statische Bibliotheken oder Objekt-Dateien.
	
	\item[-r, -{}-run]
	Wird diese Option verwendet, wird das Programm nach dem Kompilieren sofort ausgeführt.
	Für andere Architekturen als \textit{x86guest} wird dabei vorausgesetzt,
	dass die Virtualisierungssoftware \texttt{\textsc{\textbf{qemu}}} installiert ist.
	
	\item[-i, -{}-irtss-build-version]
	Mithilfe dieser Option kann man eine spezifische \textit{OctoPOS/iRTSS} Version verwenden.
	
	\item[-{}-release]
	Wird diese Option verwendet, werden Compiler-Optimierungen für Rust-Programme aktiviert.
	
	\item[-{}-fetch-irtss]
	Mithilfe dieser Option können \textit{OctoPOS/iRTSS}-builds von \\
	https://www4.cs.fau.de/invasic/octopos/ heruntergeladen werden. Hierfür wird jedoch
	eine gültige Nutzername/Passwort-Kombination in einer \texttt{\textsc{\textbf{.netrc}}}
	Datei im folgenden Format benötigt:
	\begin{verbatim}
	machine www4.cs.fau.de
	login NUTZERNAME
	password PASSWORT
	\end{verbatim}
\end{description}

Unterstützt wird das Kompilieren von eigenständigen C- und Rust-Quelldateien, als auch das Kompilieren von
Cargo-Projekten. Cargo-Projekte bieten durch das Abhängigkeitsmanagement zusätzlich den Gebrauch
von Rust-Bibliotheken, vor allem der \textit{octolib}-Bibliothek,
welche eigens für die Interaktion zwischen Rust und \textit{OctoPOS/iRTSS}
entwickelt wurde.
Daher werden sich die folgenden Prozesse primär mit dem Kompilierungsvorgang der Cargo-Projekte
beschäftigen.
Ein genauerer Überblick über die \textit{octolib} Bibliothek folgt in Kapitel \ref{sec:octolib}.

\subsection{Struktur eines invasiven Cargo-Projekts}

Wie normale Cargo-Projekte besteht ein invasives Cargo-Projekt aus mindestens einem
\texttt{\textsc{\textbf{src}}} Verzeichnis, einer Haupt-Bibliotheksdatei
und einer \texttt{\textsc{\textbf{Cargo.toml}}} Konfigurationsdatei.
Bei invasiven Cargo-Projekten muss das \texttt{\textsc{\textbf{crate-type}}} Attribut zudem immer als
\texttt{\textsc{\textbf{staticlib}}} angegeben werden,
damit das Projekt als statische Bibliothek kompiliert wird, welche dann mit \textit{OctoPOS} und dem \textit{iRTSS} 
verlinkt werden kann.

Zusätzlich muss die Haupt-Quelldatei des Projekts mit \texttt{\textsc{\textbf{\#![no\char`_std]}}} beginnen und 
anstelle der \texttt{\textsc{\textbf{main()}}} Funktion muss eine
\texttt{\textsc{\textbf{pub extern \char`"C\char`"}}} Funktion namens
\texttt{\textsc{\textbf{rust\char`_main\char`_ilet}}} als Startpunkt des Programms verwendet werden.
Diese Funktion nimmt genau einen Parameter entgegen, welcher vom Typ \texttt{\textsc{\textbf{u8}}} ist.
Über dieser Funktion muss außerdem die Anweisung \texttt{\textsc{\textbf{\#[no\char`_mangle]}}} stehen,
damit die Funktion aus einem C Kontext heraus aufrufbar ist.
Außerdem sollte die \textit{octolib} Bibliothek mit der Anweisung \texttt{\textsc{\textbf{extern crate octolib;}}}
eingebunden werden. Eine minimale Haupt-Quelldatei eines invasiven Cargo-Projekts wird in Listing 
\ref{code:minimal_cargo_source} veranschaulicht.

\begin{lstlisting}[float,caption={
Eine minimale Haupt-Quelldatei eines invasiven Cargo-Projekts
},label=code:minimal_cargo_source]
#![no_std]

extern crate octolib;

#[no_mangle]
pub extern "C" fn rust_main_ilet(claim: u8) {
}
\end{lstlisting}

\subsection{Kompilieren}

Vor der Kompilierung wird im Falle, dass \textit{leon} als Zielarchitektur ausgewählt wurde,
zuerst eine wie in Kapitel \ref{sec:rust_on_sparc} erläuterte
JSON-Spezifikationsdatei generiert,
in der der Pfad zum SPARC-V8 \texttt{\textsc{\textbf{gcc}}} Compiler automatisch eingetragen wird.
Für die anderen beiden Zielarchitekturen ist ein solcher Schritt nicht notwendig,
da diese offiziell von \texttt{\textsc{\textbf{rustc}}} und \texttt{\textsc{\textbf{cargo}}} unterstützt werden. 
Nachdem diese Spezifikationsdatei generiert wurde, wird das Projekt mithilfe von \texttt{\textsc{\textbf{cargo}}}
als statische Bibliothek kompiliert.
Wurde das Projekt erfolgreich kompiliert, wird anschließend eine minimale C Datei generiert,
welche die von \textit{OctoPOS/iRTSS} benötigte \texttt{\textsc{\textbf{main\char`_ilet}}} Funktion implementiert
und innerhalb dieser die \texttt{\textsc{\textbf{rust\char`_main\char`_ilet}}} Funktion aufruft.
Im Anschluss daran wird dann noch die \texttt{\textsc{\textbf{shutdown}}} Funktion aufgerufen, um die 
Ausführung des Programms zu beenden. Diese generierte C Datei wird mithilfe eines passenden
\texttt{\textsc{\textbf{gcc}}} Compilers als Objekt-Datei kompiliert
und anschließend mit \textit{OctoPOS/iRTSS} und der zuvor kompilierten statischen Rust-Bibliothek verlinkt.
Nun sollte eine ausführbare Datei existieren, welche auf einem Rechner mit x86-Architektur im Falle der "`x86guest"' 
Architektur nativ ausführbar ist oder im Falle von \textit{x64native} und \textit{leon} mithilfe eines Emulators wie 
\texttt{\textsc{\textbf{qemu}}}.

Im Anschluss an die Kompilierung werden,
vorausgesetzt die \texttt{\textsc{\textbf{-{}-keep}}}-Option wurde nicht verwendet,
jegliche temporäre Dateien gelöscht, beispielsweise die statische Rust-Bibliothek,
die minimale C Datei oder auch die C Objektdatei.

\section{\textit{octolib}}\label{sec:octolib}

Zusätzlich zum \textit{octorust} Programm wurde ebenfalls eine Rust-Bibliothek names \textit{octolib} geschrieben. 
Diese Bibliothek soll die Interaktion zwischen Rust und \textit{OctoPOS/iRTSS} ermöglichen und and die neue 
Programmiersprache anpassen. \textit{OctoPOS/IRTSS} bieten eine C-Schnittstelle an, welche man mithilfe von Rusts
Foreign Function Interface und der \textit{libc} Bibliothek aus Rust heraus ansprechen kann.
Die \textit{libc} Bibliothek bietet die Möglichkeit, Komponenten, welche die Standardbibliothek benötigen, 
auszulassen, daher ist sie auch auf der SPARC-V8 Architektur nutzbar.

Die \textit{octolib} Bibliothek wird vor dem Kompilieren als Abhängigkeit in die Cargo.toml Datei des
zu kompiliernden Cargo Projekts eingefügt.
Hierfür wird mithilfe des \texttt{\textsc{\textbf{toml}}} Python Moduls der derzeitige Inhalt der Cargo.toml Datei
eingelesen, der Pfad zur lokal installierten \textit{octolib} Bibliothek in diese Daten injiziert und anschließend 
wieder in die Konfigurationsdatei geschrieben.
Eine Kopie der ursprünglichen Daten wird vorerst noch im Hauptspeicher behalten und
nachdem die Kompilierung beendet wurde, wobei hier nicht unterschieden wird, ob diese erfolgreich war oder nicht,
werden die alten Daten wieder in die Cargo.toml Datei geschrieben,
um zu vermeiden dass die Cargo.toml Datei eine Konfiguration enthält, die auf eine lokale Konfiguration basiert.
Damit die Bibliothek für jedes zu kompilierende Programm auffindbar ist,
wird eine lokale Kopie während der Installation des \text-it{octorust}-Programms im
\texttt{\textsc{\textbf{~/.octorust}}} Verzeichnis erstellt.

\subsection{Struktur}

Die \textit{octolib} Bibliothek besteht aus einer Hauptbibliotheksdatei namens \texttt{\textsc{\textbf{lib.rs}}},
welche alle anderen Module innerhalb des Projekts einbindet. Diese
Datei implementiert außerdem die in Kapitel \ref{sec:rust_on_sparc} erwähnten Funktionen
\texttt{\textsc{\textbf{eh\char`_personality}}}, \texttt{\textsc{\textbf{eh\char`_unwind\char`_resume}}} und
\texttt{\textsc{\textbf{panic\char`_fmt}}}, sodass nicht jedes einzelne Projekt diese aufs Neue implementieren muss. 
Sobald die \textit{octolib} Bibliothek mit der Anweisung \texttt{\textsc{\textbf{extern crate octolib;}}}
ins Projekt eingebunden wurde, werden diese Funktionen ebenfalls importiert.

Die Bibliothek bietet die folgenden Module:

\begin{description}

	\item[bindings] 
	Dieses Modul enthält die direkten C-Rust Bindings zur C-Schnittstelle von \textit{OctoPOS/iRTSS}.
	
	\item[helper]
	Dieses Modul enthält eine Sammlung an Hilfsfunktionen, die nicht direkt mit dem \textit{OctoPOS} Betriebssystem 
	oder der \textit{iRTSS} Middleware in Verbindung stehen, aber trotzdem bei der Programmierung
	nützlich sein können.
	
	\item[improvements]
	Rust-spezifische Abstraktionen, die es dem Programmierer erleichtern, Programme mit Rust für das invasive
	Systeme zu schreiben.
	
	\item[octo\char`_structs]
	Rust-äquivalente Strukturen zu denen, die in der C-Schnittstelle verwendet werden.
	
	\item[octo\char`_types]
	Rust-äquivalente Typen zu denen, die in der C-Schnittstelle verwendet werden.

\end{description}

\subsection{Direkte C-Rust Bindings}

Im ersten Schritt wurden die Funktionen der C-Schnittstelle direkt Eins-zu-Eins als Rust-Funktionen eingebunden.
Dank der Foreign Function Interface müssen lediglich die Parameter- und Rückgabetypen
an die neue Programmiersprache angepasst werden.
Für die meisten Typen gibt es direkte Äquivalente in Rust, jegliche anderen Typen
stellt die \textit{libc} Bibliothek zur Verfügung. Ein nennenswertes Beispiel eines Typs, der nicht in Rust enthalten 
ist, ist der \texttt{\textsc{\textbf{void}}} Typ. Dieser Typ kann beispielsweise durch den
\texttt{\textsc{\textbf{c\char`_void}}} Typen aus der \textit{libc} Bibliothek emuliert werden.
\texttt{\textsc{\textbf{void}}}-Zeiger (\texttt{\textsc{\textbf{void*}}}), welche an einigen Stellen der
C-Schnittstelle verwendet werden, können so als \texttt{\textsc{\textbf{"`*mut c\char`_void}}} in Rust
dargestellt werden.

Um die Funktionen erfolgreich in die Rust-Bibliothek einbinden zu können,
muss man einen \texttt{\textsc{\textbf{extern}}}-Block erstellen und dort
die Funktionsdefinitionen vornehmen.
Damit diese auch von anderen Modulen oder Projekten aufgerufen werden können, müssen diese
als \texttt{\textsc{\textbf{pub fn}}} deklariert werden.
Da diese Funktionen lediglich importierte C-Funktionen ohne jegliche Sicherheitsgarantien
sind, muss man bei jedem Gebrauch dieser Funktionen einen \texttt{\textsc{\textbf{unsafe}}}-Block verwenden.

\subsubsection{Strukturen und Typen}

Zusätzlich zu den Funktionsdefinitionen mussten die in der C-Schnittstelle definierten
Strukturen und Typen ebenfalls portiert werden.
Dank der \texttt{\textsc{\textbf{\#[repr(C)]}}} Anweisung kann man in Rust Strukturen erstellen,
welche kompatibel mit ihren C-Äquivalenten sind.
Diese Strukturen sind teils von Platform-spezifischen Konstanten abhängig, welche man in Rust mit der
\texttt{\textsc{\textbf{\#[cfg(target\char`_arch = \"Zielarchitektur\")]}}} Anweisung
für unterschiedliche Architekturen definieren kann.
Benutzerdefinierte Typen, welche prinzipiell Aliase für bestehende Typen sind, lassen sich in Rust durch das
Schlüsselwort \texttt{\textsc{\textbf{type}}} erstellen.

\subsubsection{Das Helper-Modul}

Das \texttt{\textsc{\textbf{helper}}}-Modul enthält Adapter für C-Funktionen,
für die in Rust ohne Zugriff zur Standardbibliothek keine Alternative existieren.
So werden beispielsweise mehrere Adapter um die \texttt{\textsc{\textbf{printf}}}-Funktion geboten,
welche je eine andere Anzahl an zusätzlichen Parametern bieten, um Variablen ausdrucken zu können.

Beachtet werden muss bei diesen print-Funktionen, dass die überreichten Zeichenketten manuell
Null-terminiert werden müssen. Um "`Hello World"' zu drucken, muss man also die Anweisung
\texttt{\textsc{\textbf{print(\char`"Hello World\char`\\n\char`\\0\char`");}}} verwenden.

\subsection{Rust-spezifische Verbesserungen}

Im Folgenden werden verschiedene Abstraktionen über die direkten C-Rust Bindings erstellt, welche das Programmieren
vereinfachen und die Stärken von Rust ausnutzen sollen.

\subsubsection{Constraints}

Die \texttt{\textsc{\textbf{Constraints}}} Struktur ist eine leichte,
objektorientierte Abstraktion für die \textit{Constraints}, mit denen die Anzahl Ressourcen,
die einem \textit{Claim} zur Verfügung stehen sollen, spezifiziert werden.
Die Implementierung dieser Struktur bietet einen Konstruktor, welcher eine neue
\texttt{\textsc{\textbf{constraints\char`_t}}} Struktur mithile der
\texttt{\textsc{\textbf{agent\char`_constr\char`_create}}}
Funktion aus dem \texttt{\textsc{\textbf{octo\char`_bindings}}} Modul erstellt und anschließend einige Standardwerte setzt. 
Dem Konstruktor werden zudem 2 Parameter übergeben,
welche die minimale und maximale Anzahl an Rechenelementen spezifizieren.

Alle Parameter der \textit{Constraints} lassen sich nachträglich durch Methoden der Struktur ändern.
Hierbei profitieren vor allem die Methoden, welche boolesche Werte als Parametertypen besitzen,
denn diese werden in den direkten C-Rust Bindings statt mit \texttt{\textsc{\textbf{bool}}}-Werten mit
\texttt{\textsc{\textbf{u8}}}-Werten aufgerufen. Dies liegt daran, dass es in C keinen
\texttt{\textsc{\textbf{bool}}} oder ähnlichen Typen gibt, dort werden stattdessen Zahlenwerte auf den
Wert \texttt{\textsc{\textbf{0}}} geprüft. Dies wird in den Methoden dieser Struktur jedoch abstrahiert,
so dass man in Rust den \texttt{\textsc{\textbf{bool}}}-Typ verwenden kann.

Außerdem unterstützt die Methode \texttt{\textsc{\textbf{merge\char`_constraints}}} anstelle einer
\texttt{\textsc{\textbf{constraints\char`_t}}} Struktur aus der C-Schnittstelle eine in Rust definierte
\texttt{\textsc{\textbf{Constraints}}} Struktur. Möchte man trotzdem lieber eine
\texttt{\textsc{\textbf{constraints\char`_t}}}-Struktur verwenden, so kann stattdessen die
\texttt{\textsc{\textbf{merge\char`_constraints\char`_t}}} Methode verwendet werden.

Um Zugriff auf die interne \texttt{\textsc{\textbf{constraints\char`_t}}} Struktur zu erhalten, kann die
\texttt{\textsc{\textbf{to\char`_constraints\char`_t}}} Methode aufgerufen werden.
Hiernach ist die \texttt{\textsc{\textbf{Constraints}}}-Struktur selbst nicht mehr verwendbar,
da die interne \texttt{\textsc{\textbf{constraints\char`_t}}}-Variable aufgrund der \textit{Move-Semantik}
nicht mehr im Besitz des zugehörigen Speicherbereichs ist.

\subsubsection{Closures}

Rust Closures können nicht ohne Weiteres einer C-Schnittstelle als Parameter übergeben werden.
Somit ist es mit den direkten C-Rust Bindings nicht möglich, Closures zu verwenden,
um Code auf den Rechenelementen ausführen zu lassen. Dies wäre jedoch eine
wünschenswerte Funktionalität. Da Closures aus der Funktion selbst als auch ihrem
Erstellungskontext bestehen\cite{embeddedRustOS}, ist es nicht möglich,
eine Closure direkt als eine normale Rust Funktion zu verwenden.
Stattdessen erstellt man einen Zeiger auf die Datenregion,
welche die Closure repräsentiert und übergibt diesen dann einer
\texttt{\textsc{\textbf{extern \char`"C\char`"}}} Funktion, welche aus diesem Zeiger die Closure zurückkonvertiert.
Diese \texttt{\textsc{\textbf{extern \char`"C\char`"}}}-Funktion kann dann der C-Schnittstelle zusammen mit dem
Zeiger auf die Closure-Daten übergeben werden.

Die Erstellung des Closure-Zeigers wird direkt in der später genauer erläuterten
\texttt{\textsc{\textbf{infect}}}-Methode der \texttt{\textsc{\textbf{AgentClaim}}}-Struktur
erledigt, während zur Rückkonvertierung eine \texttt{\textsc{\textbf{extern \char`"C\char`"}}}-Funktion namens \texttt{\textsc{\textbf{execute\char`_closure}}} erstellt wurde.

\subsubsection{AgentClaim}

Die größte Abstraktion in der \textit{octolib} Bibliothek ist die \texttt{\textsc{\textbf{AgentClaim}}} Struktur.
Diese bietet zunächst einmal eine Abstraktionsschicht über die \texttt{\textsc{\textbf{agentclaim\char`_t}}}
Struktur aus der C-Schnittstelle und verschiedene Methoden um diese zu verwenden.

Der Konstruktor nimmt als einzigen Parameter eine \texttt{\textsc{\textbf{Constraints}}}-Struktur entgegen,
welche verwendet wird, um die Ressourcen des \textit{Claims} zu spezifizieren.
Mit diesen \textit{Constraints} wird dann eine interne
\texttt{\textsc{\textbf{agentclaim\char`_t}}}-Struktur initialisiert,
welche für alle folgenden Methodenaufrufe verwendet wird. Es wird also bereits im Konstruktor implizit die
\textit{Invade}-Phase des invasiven Computing ausgeführt.

Eine praktische Methode fürs Debugging ist \texttt{\textsc{\textbf{set\char`_verbose}}}.
Diese Methode erlaubt es, die Ausführlichkeit der auf die Kommandozeile gedruckten Informationen der
\texttt{\textsc{\textbf{AgentClaim}}}-Struktur zu beeinflussen.

Die \texttt{\textsc{\textbf{reinvade}}}-Methode elaubt es, eine \textit{Reinvade}-Operation auf dem
\textit{Claim} auszuführen. Diese erlaubt es, die von der internen
\texttt{\textsc{\textbf{agentclaim\char`_t}}}-Struktur verwendeten \textit{Constraints} anzupassen.
Diese Methode nimmt einen optionalen Parameter an,
mit dem man neue \textit{Constraints} für den \textit{Claim} spezifizieren kann.
Werden neue \textit{Constraints} gesetzt, werden die alten \textit{Constraints} mit der
\texttt{\textsc{\textbf{agent\char`_constr\char`_delete}}}-Funktion aus dem Speicher gelöscht. 

Die wohl wichtigste Methode der \texttt{\textsc{\textbf{AgentClaim}}}-Struktur ist die
\texttt{\textsc{\textbf{infect}}}-Methode. Mit dieser kann eine invasive \textit{Infect}-Operation
auf dem \textit{Claim} ausgeführt werden.
Dieser Methode wird eine Funktion oder eine Closure als Parameter übergeben,
welche dann auf den Rechenelementen des \textit{Claims} ausgeführt wird.
Zusätzlich können ebenfalls Parameterdaten für die einzelnen Recheneinheiten übergeben werden.
Diese Parameterdaten werden als ein Array von \texttt{\textsc{\textbf{void}}}-Zeigern übergeben.
Die Anzahl der Elemente des Arrays muss mit der Anzahl von vom \textit{Claim} reservierten Rechenelemente 
übereinstimmen. Bei einer zu geringen Anzahl an Rechenelementen kann es ansonsten zu Fehlern kommen.
Bei einer zu großen Zahl an Rechenelementen kommt es zum Pufferüberlauf, welches den Absturz des Programms zur
Folge hat.

In der \texttt{\textsc{\textbf{infect}}}-Methode werden Signal-Strukturen aus der C-Schnittstelle verwendet,
um die Kommunikation zwischen verschiedenen Rechenelementen zu verwalten.
Um dies zu vereinfachen, wird die übergebene Funktion oder Closure nochmals von einer Closure umgeben,
welche eine Referenz auf eine in der \texttt{\textsc{\textbf{infect}}}-Methode erstellten Signal-Struktur besitzt
und über dieses Signal signalisiert, wann diese Closure die Ausführung ihrer Aufgabe beendet hat.
Hierdurch ist es möglich, nach der Initialisierung der \textit{i-lets} auf die
vollständige Abarbeitung dieser zu warten oder alternativ das Signal, mit dem die \textit{i-lets} mit dem
Hauptfaden des Programms kommunizieren, als Rückgabewert zu verwenden.
Um beide dieser Varianten zu bieten, gibt es zusätzlich zur \texttt{\textsc{\textbf{infect}}}-Methode,
welche auf die vollständige Ausführung der \textit{i-lets} wartet,
die \texttt{\textsc{\textbf{infect\char`_async}}}-Methode,
welche die \texttt{\textsc{\textbf{simple\char`_signal}}}-Struktur,
mit der die \textit{i-lets} das Ende ihrer Ausführung signalisieren,
als Rückgabewert verwendet. Anschließend kann der Nutzer mit der 
\texttt{\textsc{\textbf{simple\char`_signal\char`_wait}}}-Funktion aus dem
\texttt{\textsc{\textbf{octo\char`_bindings}}}
Modul auf die vollständige Ausführung aller \textit{i-lets} warten.
Die interne Closure, welche die Kommunikation über Signale implementiert, wird innerhalb der
\texttt{\textsc{\textbf{infect}}} Methode zu einem \texttt{\textsc{\textbf{void}}}-Zeiger konvertiert,
welcher den \textit{i-lets} anschließend als Datenparameter übergeben wird.

Für jedes Rechenelement wird iteriert und ein \textit{i-let} initialisiert.
Diese \textit{i-lets} erhalten die \texttt{\textsc{\textbf{execute\char`_closure}}}-Funktion als ihren
Funktionsparameter und den Closure-Zeiger als ihren ersten Datenparameter.
Wurden der \texttt{\textsc{\textbf{infect}}}-Methode zusätzliche Datenparameter
übergeben, so werden diese als zweite Datenparameter den \textit{i-lets} übergeben.
Nachdem ein \text{i-let} initialisiert wurde, wird die eigentliche \textit{Infect}-Operation durchgeführt.

Das bereits im Grundlagenkapitel erläuterte \textit{Drop}-Trait für Rust-Strukturen wird für die
\texttt{\textsc{\textbf{AgentClaim}}}-Struktur implementiert.
Verlässt eine Instanz der \texttt{\textsc{\textbf{AgentClaim}}}-Struktur den Geltungsbereich,
wird eine \texttt{\textsc{\textbf{Retreat}}}-Operation auf der
internen \texttt{\textsc{\textbf{agentclaim\char`_t}}}-Struktur ausgeführt und zusätzlich
die \textit{Constraints} des \textit{Claims} gelöscht. Dieses implizite \textit{Retreat}
stellt sicher, dass die vom Claim verwendeten Ressourcen nach dem Gebrauch wieder freigegeben werden und
somit anderen Programmteilen zur Verfügung stehen.
