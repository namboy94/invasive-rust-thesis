\chapter{Evaluation}\label{sec:eval}

Im Folgenden wird der Gebrauch von Rust im Zusammenhang mit dem invasiven Rechnen evaluiert. Hierbei
wird vor allem mit den bereits unterstützten Sprachen C und X10 verglichen.

Jedes der folgenden Programme wurde auf der folgenden Hardware/Software Konfiguration kompiliert und ausgeführt:

\begin{table}[hb]
	\begin{center}
		\begin{tabular}{ll}
			\midrule
			CPU & Intel Core i5-5200U @ 2.2GHZ x 2 \\
			Hauptspeicher & 8GB \\
			Betriebssystem & Ubuntu 16.04.3 LTS \\
			Linux Kernel & Version 4.4.0-97-generic \\
			gcc(x86) & 5.4.0 \\
			gcc(SPARC-V8) & 7.2.0 \\
			iRTSS & 2017-06-07-nightly, x86guest-generic \\
			rustc & 1.19.0-nightly \\
			JDK & openjdk 1.8.0\char`_131 \\
			octorust & Version 1.0.0 \\
			x10i & Commit 31183335a89917f489046da746c5181174a7bdb3 \\
			\bottomrule
		\end{tabular}
	\end{center}
	\caption{
		Dies ist die Hard- und Software Konfiguration des Rechners,
		auf dem die nachfolgenden Programme kompiliert und ausgeführt wurden.
	}
	\label{fig:specs_table}
\end{table}

Für jedes betrachtete Programm wurden je zwei Versionen pro Programmiersprache betrachtet.
Zum einen mit Compiler-Optimierungen,
zum anderen ohne. Bei Messungen der Ausführungszeit wurde die Wall-Clock-Time in Sekunden als Messwert verwendet.
Jedes Programm, dessen Laufzeit
überprüft wird, wurde mithilfe des \texttt{\textsc{\textbf{temci}}}-Benchmarkprogramms fünfzig mal ausgeführt.
C- und Rust-Programme wurden mit \textit{octorust} kompiliert, X10-Programme mit dem \textit{x10i} Compiler.
Alle berechneten Werte wurden auf vier Nachkommastellen gerundet.

Um die statistische Relevanz der Ergebnisse zu evaluieren, wurden zu jeder Messung der Laufzeit
das arithmetische Mittel (Abbildung \ref{formula:mean}),
der Median (Abbildung \ref{formula:median}),
die Stichproben-Standardabweichung (Abbildung \ref{formula:std_deviation}) und
der zugehörige Stichproben-Variationskoeffizient (Abbildung \ref{formula:variation_coefficient}) berechnet
\cite{wtSkript}.

\begin{figure}
	\caption{Das Arithmetische Mittel}
	\begin{equation*}
		\bar{x} = \frac{1}{n} \sum_{i=1}^n{x_i}
	\end{equation*}
	\label{formula:mean}
\end{figure}

\begin{figure}
	\caption{Der Median}
	\begin{equation*}
		\tilde{x} =
		\begin{cases}
			x_{\frac{n + 1}{2}} & n \text{ ungerade} \\
			\frac{1}{2}(x_\frac{n}{2} + x_{\frac{n}{2} + 1}) & n \text{ gerade} \\
		\end{cases}
	\end{equation*}
	\label{formula:median}
\end{figure}

\begin{figure}
	\caption{Die Stichproben-Standardabweichung}
	\begin{equation*}
		s_x = \sqrt{\frac{1}{n - 1} \sum_{i=1}^n{(x_i - \bar{x})^2}}
	\end{equation*}
	\label{formula:std_deviation}
\end{figure}

\begin{figure}
	\caption{Der Stichproben-Variationskoeffizient für Werte $x_i > 0$}
	\begin{equation*}
		v_x = \frac{s_x}{\bar{x}}
	\end{equation*}
	\label{formula:variation_coefficient}
\end{figure}



\section{Kompilierungsdauer und Dateigröße}\label{sec:compile_time_filesize}

Zu Beginn werden die Kompilierungsdauer und die Dateigröße bei Gebrauch der einzelnen Programmiersprachen
verglichen. Diese Werte wurden für ein simples Programm betrachtet, welches lediglich die Ausführung startet und
anschließend sofort wieder beendet.

\begin{table}[hb]
	\begin{center}
		\begin{tabular}{lrrrr}
			\toprule
			Sprache    & $\bar{x}$ & $\tilde{x}$ & $s_x$ & $v_x$\\
			\midrule
			C          &   0.4924 &   0.4900 & 0.0170 & 3.4465\% \\
			C (opt)    &   0.5586 &   0.5600 & 0.0134 & 2.3993\% \\
			Rust       &   0.8174 &   0.8200 & 0.0164 & 2.0051\% \\
			Rust (opt) &   0.8116 &   0.8100 & 0.0136 & 1.6765\% \\
			X10        &  82.8640 &  83.0750 & 1.3321 & 1.6076\% \\
			X10 (opt)  & 192.1182 & 191.9150 & 1.4026 & 0.7301\% \\
			\bottomrule
		\end{tabular}
	\end{center}
	\caption{
		Diese Tabelle stellt die Messergebnisse der Kompilierungsdauer des \textit{startup}-Programms dar.
		Alle Werte sind auf vier Nachkommastellen gerundet und mit Ausnahme des Variationskoeffizienten
		in Sekunden angegeben.
	}
	\label{fig:compile_table}
\end{table}

\begin{table}[hb]
	\begin{center}
		\begin{tabular}{lr}
			\toprule
			Sprache & Dateigröße \\
			\midrule
			C          & 10117708 \\
			C (opt)    & 10117708 \\
			Rust       & 10118632 \\
			Rust (opt) & 10117760 \\
			X10        & 16630348 \\
			X10 (opt)  & 17546444 \\
			\bottomrule
		\end{tabular}
	\end{center}
	\caption{
		Diese Tabelle stellt die Messergebnisse der Dateigrößen des \textit{startup}-Programms dar.
		Die Messwerte sind in der Anzahl Bytes, welche die ausführbare Datei ausmachen, angegeben.
	}
	\label{fig:filesize_table}
\end{table}

Die Ergebnisse in Tabelle \ref{fig:compile_table} zeigen, dass X10-Programme im Vergleich zu C- oder Rust-Programmen
eine ungefähr 100-fache Kompilierungsdauer vorweisen. Mit aktivierten Compiler-Optimierungen steigt diese
Zahl auf über das 200-fache. Zudem sind die Dateigrößen des X10-Programme mindestens 6511716 Byte
größer als die der C- und Rust-Programme.

\section{Laufzeitverhalten}

Um die erreichbare Leistung der Programmiersprachen zu vergleichen,
wurden im Folgenden unterschiedliche Programme entwickelt.
Diese Programme wurden soweit möglich in allen Programmiersprachen einheitlich implementiert und ermöglichen so,
das Laufzeitverhalten der Programmiersprachen miteinander zu vergleichen.

\subsection{Vergleich der Anlaufzeit}

Es wird zunächst überprüft, wie lange ein Programm,
welches in einer zu betrachtenden Programmiersprachen geschrieben wurde,
benötigt, um die Ausführung zu starten und anschließend sofort wieder zu beenden.
Es wurde dasselbe Programm verwendet, welches in Kapitel \ref{sec:compile_time_filesize} zur Messung
der Kompilierungsdauer und Dateigröße verwendet wurde.

\begin{table}[hb]
	\begin{center}
		\begin{tabular}{lrrrr}
			\toprule
			Sprache    & $\bar{x}$ & $\tilde{x}$ & $s_x$ & $v_x$ \\
			\midrule
			C          & 0.3772 & 0.3700 & 0.0140 &  3.7123\% \\
			C (opt)    & 0.3852 & 0.3800 & 0.0171 &  4.4268\% \\
			Rust       & 0.3820 & 0.3800 & 0.0139 &  3.6258\% \\
			Rust (opt) & 0.4026 & 0.3800 & 0.0459 & 11.4080\% \\
			X10        & 1.7738 & 1.7800 & 0.0255 &  1.4357\% \\
			X10 (opt)  & 1.7506 & 1.7500 & 0.0234 &  1.3380\% \\
			\bottomrule
		\end{tabular}
	\end{center}
	\caption{
		Diese Tabelle stellt die Messergebnisse der Laufzeit des \textit{startup}-Benchmarks dar.
		Dieser Benchmark simuliert das Starten und sofortige Schließen eines Programms.
		Alle Werte sind auf vier Nachkommastellen gerundet und mit Ausnahme des Variationskoeffizienten
		in Sekunden angegeben.
	}
	\label{fig:startup_table}
\end{table}

Anhand der Werte in Tabelle \ref{fig:startup_table} kann man prinzipiell erkennen,
dass die Anlaufzeiten von C und Rust sich sehr nahe sind, etwaige Unterschiede lassen sich auf die Varianz der
Ergebnisse zurückführen. Die Implementierung in X10 benötigt jedoch im Schnitt ungefähr 1,3 Sekunden mehr Zeit
als es bei den beiden anderen Programmiersprachen der Fall ist.


\subsection{Berechnen von Primzahlen}\label{sec:primes_calc}

Um die Rechenleistung der verschiedenen Programmiersprachen bei einem intensiveren mathematischen Problem zu 
vergleichen, wurden Programme geschrieben, welche Primzahlen berechnen.
Hierbei wurden zwei unterschiedliche Ansätze verwendet. Zum einen wurde
eine naive Berechnung implementiert, welche jede Zahl individuell auf Teilbarkeit mit kleineren Zahlen prüft. 
Andererseits wurde das Sieb von Eratosthenes verwendet, welches eine effiziente Methode zum Berechnen von Primzahlen
ist.

Im Falle des Siebs von Eratosthenes kann Rust in diesem Kontext keine Quadratwurzeloperation durchführen,
da diese die Standardbibliothek benötigt. Daher wurden in allen Sprachen anstelle einer Quadratwurzelfunktion
feste Zahlen verwendet, um den Vergleich zwischen den Sprachen gerecht zu gestalten.
Da jedoch nur eine Quadratwurzeloperation für das Sieb des Eratosthenes
benötigt wird, wäre dies ohnehin aller Wahrscheinlichkeit nach kein entscheidender Faktor bei der Laufzeitmessung.

\begin{table}[hb]
	\begin{center}
		\begin{tabular}{lrrrr}
			\toprule
			Sprache    & $\bar{x}$ & $\tilde{x}$ & $s_x$ & $v_x$   \\
			\midrule
			C          &  4.6898  & 4.6950  & 0.0313 & 0.6681\% \\
			C (opt)    &  4.2134  & 4.2150  & 0.0297 & 0.7041\% \\
			Rust       & 61.6462  & 61.6050 & 0.2849 & 0.4621\% \\
			Rust (opt) &  4.9692  &  4.9650 & 0.0319 & 0.6413\% \\
			X10        & 10.5830  & 10.5500 & 0.1368 & 1.2923\% \\
			X10 (opt)  &  5.9694  &  5.9700 & 0.0719 & 1.2039\% \\
			\bottomrule
		\end{tabular}
	\end{center}
	\caption{
		Diese Tabelle stellt die Ergebnisse des \textit{naive-primes}-Benchmarks dar.
		Dieser Benchmark überprüft die ersten \texttt{\textsc{\textbf{50000}}} natürlichen Zahlen darauf,
		ob es sich bei ihnen um Primzahlen handelt,
		indem er für jede Zahl individuell alle kleinere Zahlen auf Teilbarkeit prüft.
		Alle Werte sind auf vier Nachkommastellen gerundet und mit Ausnahme des Variationskoeffizienten
		in Sekunden angegeben.
	}
	\label{fig:primes_naive_table}
\end{table}

\begin{table}[hb]
	\begin{center}
		\begin{tabular}{lrrrr}
			\toprule
			Sprache    & $\bar{x}$ & $\tilde{x}$ & $s_x$ & $v_x$ \\
			\midrule
			C          & 0.6250 & 0.6150 & 0.0196 & 3.1382\% \\
			C (opt)    & 0.5584 & 0.5500 & 0.0175 & 3.1408\% \\
			Rust       & 0.8362 & 0.8300 & 0.0138 & 1.6544\% \\
			Rust (opt) & 0.4984 & 0.4900 & 0.0210 & 4.2196\% \\
			X10        & 2.3508 & 2.3400 & 0.0347 & 1.4782\% \\
			X10 (opt)  & 1.9892 & 1.9900 & 0.0274 & 1.3771\% \\
			\bottomrule
		\end{tabular}
	\end{center}
	\caption{
		Diese Tabelle stellt die Ergebnisse des \textit{eratosthenes-primes}-Benchmarks dar.
		Dieser Benchmark überprüft mithilfe des Siebs von Eratosthenes die ersten \texttt{\textsc{\textbf{1690000}}}
		natürlichen Zahlen darauf, ob es sich bei ihnen um Primzahlen handelt.
		Da Rust ohne die Standardbibliothek keine Quadratwurzeloperationen
		unterstützt, wurde in jeder Implementierung der Wert \texttt{\textsc{\textbf{1300}}} fest als das Ergebnis
		der Quadratwurzeloperation einprogrammiert.
		Die gegebenen Werte wurden gewählt, da höhere Werte in Rust und C einen Stack-Überlauf zur Folge hätten.
		Alle Werte sind auf vier Nachkommastellen gerundet und mit Ausnahme des Variationskoeffizienten
		in Sekunden angegeben.
	}
	\label{fig:primes_eratosthenes_table}
\end{table}

Mit den Werten in den Tabellen \ref{fig:primes_naive_table} und \ref{fig:primes_eratosthenes_table} lässt sich keine
eindeutige Aussage darüber treffen, welche der Programmiersprachen die Aufgabe am effizientesten
gelöst hat, vorausgesetzt, es werden die Versionen mit Compiler-Optimierungen betrachtet. Im Schnitt weisen C und Rust
jedoch eine 1-1,8 Sekunden bessere Laufzeit als X10 auf.
Allerdings muss dabei auch die längere Anlaufzeit von X10 Programmen in Betracht gezogen werden.
Zieht man diese ab, sind die Ergebnisse zu nah aneinander, als dass man eine klare Tendenz beobachten könne.

Werden allerdings die Versionen ohne Compiler-Optimierungen betrachtet, gibt es deutliche Unterschiede zwischen den
Sprachen. In Tabelle \ref{fig:primes_naive_table} ist beispielsweise zu erkennen, dass die unoptimierte
Version des Rust-Programms ungefähr 6-mal langsamer als die nächstlangsamste betrachtete Konfiguration ist.

\FloatBarrier

\subsection{Paralleles Berechnen von Primzahlen}

Um auch die invasiven Aspekte der Sprachen zu evaluieren, wurden X10-, C- und Rust-Programme geschrieben,
welche Primzahlen in einer parallelen Art und Weise berechnen.
Hierfür wird die Menge an Zahlen, welche überprüft werden sollen,
ob es sich bei ihnen um Primzahlen handelt, gleichermaßen in Partitionen getrennt.
Die Anzahl und Größe der Partitionen hängen hierbei von der Anzahl an verfügbaren Rechenelementen ab.
Jedes der Rechenelemente berechnet anschließend die ihm zugewiesene Menge an Zahlen naiv,
indem jede Zahl mit jeder kleineren Zahl auf Teilbarkeit geprüft wird.

\begin{table}[hb]
	\begin{center}
		\begin{tabular}{lrrrr}
			\toprule
			Sprache & $\bar{x}$ & $\tilde{x}$ & $s_x$ & $v_x$ \\
			\midrule
			C          &  33.9812 &  33.7450 & 0.4918 & 1.4473\% \\
			C (opt)    &  29.9782 &  29.9750 & 0.0377 & 0.1257\% \\
			Rust       & 476.9242 & 475.8950 & 3.6647 & 0.7684\% \\
			Rust (opt) &  37.6212 &  37.3300 & 0.6651 & 1.7679\% \\
			X10        &  67.9618 &  67.4050 & 1.1505 & 1.6929\% \\
			X10 (opt)  &  31.4584 &  31.3050 & 0.4181 & 1.3290\% \\
			\bottomrule
		\end{tabular}
	\end{center}
	\caption{
		Diese Tabelle stellt die Laufzeitergebnisse des sequentiellen Ergebnisses des
		\textit{parallel-primes}-Benchmarks dar,
		es wurde also die zu verwendende Anzahl an Rechenelementen auf \texttt{\textsc{\textbf{1}}} gesetzt.
		Dieser Benchmark überprüft die ersten \texttt{\textsc{\textbf{500000}}} natürlichen Zahlen darauf,
		ob es sich bei ihnen um Primzahlen handelt.
		Alle Werte sind auf vier Nachkommastellen gerundet und mit Ausnahme des Variationskoeffizienten
		in Sekunden angegeben.
	}
	\label{fig:primes_parallel_one_table}
\end{table}

\begin{table}[hb]
	\begin{center}
		\begin{tabular}{lrrrr}
			\toprule
			Sprache & $\bar{x}$ & $\tilde{x}$ & $s_x$ & $v_x$ \\
			\midrule
			C          &  16.4720 &  16.4600 & 0.0528 & 0.3203\% \\
			C (opt)    &  13.8018 &  13.7950 & 0.0352 & 0.2551\% \\
			Rust       & 242.6560 & 242.4800 & 0.7602 & 0.3133\% \\
			Rust (opt) &  17.8830 &  17.8700 & 0.0862 & 0.4823\% \\
			X10        &  46.3264 &  46.3800 & 0.2348 & 0.5069\% \\
			X10 (opt)  &  16.8028 &  16.7600 & 0.1181 & 0.7026\% \\
			\bottomrule
		\end{tabular}
	\end{center}
	\caption{
		Diese Tabelle stellt die Laufzeitergebnisse des \textit{parallel-primes}-Benchmarks dar,
		wenn \texttt{\textsc{\textbf{8}}} Rechenelemente verwendet werden.
		Dieser Benchmark überprüft die ersten \texttt{\textsc{\textbf{500000}}} natürlichen Zahlen darauf,
		ob es sich bei ihnen um Primzahlen handelt.
		Alle Werte sind auf vier Nachkommastellen gerundet und mit Ausnahme des Variationskoeffizienten
		in Sekunden angegeben.
	}
	\label{fig:primes_parallel_eight_table}
\end{table}

Der Vergleich der Werte in Tabelle \ref{fig:primes_parallel_one_table} und \ref{fig:primes_parallel_eight_table} 
lässt, anders wie in Kapitel \ref{sec:primes_calc}, Aussagen zur Laufzeiteffizienz der einzelnen Sprachen zu.
Vor allem in Tabelle \ref{fig:primes_parallel_one_table} kann man erkennen, dass C mit 30 Sekunden die beste Laufzeit aufweist,
gefolgt von X10 mit 31 Sekunden. Die schlechteste Laufzeit wies Rust auf mit 38 Sekunden auf.

Zur Bewertung von parallelen Algorithmen verwendet man unter Anderem die Werte des \textit{Speedup}
und der \textit{Effizienz}.
Den relativen \textit{Speedup} eines parallelen Programms berechnet man mit der Formel aus Abbildung\ref{formula:speedup} und die
\textit{Effizienz} mit der Formel aus Abbildung \ref{formula:effizienz}. $S(n)$ bezeichnet hierbei den \textit{Speedup},
$E(n)$ die \textit{Effizienz} und $T(x)$ die Ausführungszeit mit x Prozessoren.

\begin{figure}
	\caption{Der Speedup eines parallelen Algorithmus mit $n$ Rechenelementen}
	\begin{equation*}
		S(n) = \frac{T(1)}{T(n)}
	\end{equation*}
	\label{formula:speedup}
\end{figure}

\begin{figure}
	\caption{Die Effizienz eines parallelen Algorithmus mit $n$ Rechenelementen}
	\begin{equation*}
		E(n) = \frac{S(n)}{n}
	\end{equation*}
	\label{formula:effizienz}
\end{figure}

Mit den Werten aus Tabelle \ref{fig:primes_parallel_one_table} und \ref{fig:primes_parallel_eight_table} und diesen
Formeln kann man nun auch den Speedup und die Effizienz für das implementierte Programm für die unterschiedlichen
Varianten berechnen.

\begin{table}[hb]
	\begin{center}
		\begin{tabular}{lrrrr}
			\toprule
			Sprache & $S_{\bar{x}}(8)$ & $S_{\tilde{x}}(8)$ & $E_{\bar{x}}(8)$ & $E_{\tilde{x}}(8)$ \\
			\midrule
			C          & 2.0630 & 2.0501 & 0.2579 & 0.2563 \\
			C (opt)    & 2.1721 & 2.1729 & 0.2715 & 0.2716 \\
			Rust       & 1.9654 & 1.9626 & 0.2457 & 0.2453 \\
			Rust (opt) & 2.1037 & 2.0890 & 0.2630 & 0.2611 \\
			X10        & 1.4670 & 1.4533 & 0.1834 & 0.1817 \\
			X10 (opt)  & 1.8722 & 1.8678 & 0.2340 & 0.2335 \\
			\bottomrule
		\end{tabular}
	\end{center}
	\caption{
		In dieser Tabelle werden die berechneten Werte für den \textit{Speedup} und die \textit{Effizienz}
		des \textit{parallel-primes-Benchmark} für acht Recheneinheiten veranschaulicht.
		Hierbei wurden die Ergebnisse sowohl für den
		Median als auch für das arithmetischen Mittel berechnet. Alle Werte sind auf vier Nachkommastellen gerundet.
	}
	\label{fig:primes_parallel_speedup_efficiency_table}
\end{table}

Betrachtet man die Ergebnisse der \textit{Speedup}- und \textit{Effizienz}berechnung in Tabelle
\ref{fig:primes_parallel_speedup_efficiency_table}, so kann man keine klare Tendenz zwischen Rust und C feststellen.
Es kann also diesbezüglich nicht behauptet werden, dass eine der Sprachen durch die Parallelisierung stärker profitiert.
Die X10-Programme weisen jedoch in jedem Fall einen um mindestens 0,2 geringeren Speedup auf und auch die Effizienz ist mindestens
um 0,3 geringer als die Ergebnisse der anderen Programmiersprachen. Es kann also behauptet werden, dass Rust und C in diesem Fall
stärker von der Parallelisierung profitieren als X10

Der \textit{Speedup} ist mit einem Wert von rund 1,5 bis 2,2 in diesem Beispiel eher gering,
dies liegt jedoch am verwendeten Algorithmus,
der durch die gleichmäßige Partitionierung der betrachteten Zahlen und der naiven Berechnung
Leistungseinbußen in Kauf nimmt. Dies liegt vor allem daran, dass bei der naiven Berechnung größere Zahlen
einen höheren Rechenaufwand erfordern. Da jedoch alle Partitionen gleich groß sind, wird der \textit{i-let} mit
den höchsten Werten länger als alle anderen \textit{i-lets} benötigen, um seine Ausführung zu beenden.

\FloatBarrier

\subsection{Allokation auf dem Heap}

X10 verwaltet den Speicher mithilfe eines \textit{Garbage Collectors}, wobei C und Rust auf einen solchen verzichten.
Während \textit{Garbage Collector} ein sehr hilfreiches Werkzeug sind,
um den Programmieraufwand zu verringern, so kann dies
allerdings auch auf Kosten der Laufzeiteffizienz und Verfügbarkeit durch \textit{Garbage Collector}-Pausen geschehen.

Um zu Prüfen, ob diese Gegebenheit einen messbaren Effekt auf die Laufzeit eines Programmes haben kann,
wurde ein Programm geschrieben, welches kontinuierlich Objekte auf dem Heap erstellt,
welche anschließend wieder aus dem Geltungsbereich verschwinden.
In C muss der Speicher, in dem diese Objekte gespeichert werden, manuell mit \texttt{\textsc{\textbf{free}}}
befreit werden, in Rust wird dieser Speicher automatisch wieder befreit sobald ihr \textit{Owner} den
Geltungsbereich verlässt und im Falle von X10 werden diese Objekte vom \textit{Garbage Collector} verwaltet.

Die erstellten Objekte sind in diesem Fall Arrays von Integer-Zahlenwerten. In C und X10 wurden die
\texttt{\textsc{\textbf{int}}}-Typen, in Rust der \texttt{\textsc{\textbf{i32}}} Typ verwendet. 
Das Äquivalent zu Arrays in X10 sind Rails, daher wurden in der X10 Implementierung diese verwendet.

\begin{table}[hb]
	\begin{center}
		\begin{tabular}{lrrrr}
			\toprule
			Sprache    & $\bar{x}$ & $\tilde{x}$ & $s_x$ & $v_x$ \\
			\midrule
			C          &  19.0216 &  19.0000 & 0.0743 & 0.3904\%  \\
			C (opt)    &   0.3854 &   0.3700 & 0.0206 & 5.3513\%  \\
			Rust       &  29.5478 &  29.4800 & 0.2671 & 0.9038\%  \\
			Rust (opt) &   5.1024 &   5.1000 & 0.0258 & 0.5048\%  \\
			X10        & 339.3450 & 338.1900 & 2.3484 & 0.6920\%  \\
			X10 (opt)  & 227.8816 & 226.0900 & 2.8234 & 1.2390\%  \\
			\bottomrule
		\end{tabular}
	\end{center}
	\caption{
		Diese Tabelle stellt die Ergebnisse des \textit{garbageonly}-Benchmarks dar. Dieser Benchmark erstellt insgesamt
		\texttt{\textsc{\textbf{1000000}}} \texttt{\textsc{\textbf{int}}}/\texttt{\textsc{\textbf{i32}}}-Arrays/Rails
		mit einer Größe von \texttt{\textsc{\textbf{5000}}} Elementen.
		Alle Werte sind auf vier Nachkommastellen gerundet und mit Ausnahme des Variationskoeffizienten
		in Sekunden angegeben.
	}
	\label{fig:garbageonly_table}
\end{table}

Wie man den Ergebnissen der Tabelle \ref{fig:garbageonly_table} entnehmen kann,
benötigt X10 im Vergleich zu Rust oder C ein Vielfaches an Ausführungszeit, um die selbe Anzahl an gleich großen
Objekten zu erstellen. Betrachtet man die Ergebnisse der unoptimierten Versionen der Programme, benötigt X10
mehr als 11-mal so lange wie Rust und ungefähr 17-mal so lange wie C.
Dies könnte eine Konsequenz des \textit{Garbage Collectors} sein, es lässt sich jedoch ohne eine genaue Messung des
Speicherverhaltens keine eindeutige Aussage diesbezüglich treffen.

Bei häufiger Erstellung von neuen Objekten weisen Rust und C durch die gemessene Laufzeit
einen nennenswerten Vorteil gegenüber X10 auf. Im Falle von Rust muss der Programmierer obendrein
den Speicher nicht selbst wieder freigeben, denn dies geschieht implizit sobald eine Variable
den Geltungsbereich verlässt. Somit weist Rust gegenüber C ebenfalls trotz der ungefähr 10 Sekunden schlechteren Laufzeit
einen Vorteil auf.

\section{Aufwand des Programmierens}

Im Folgenden wird der Aufwand des Erstellens eines invasiven Programms in jeder der betrachteten
Programmiersprachen verglichen.

\subsection{Projektstruktur}

Zunächst wird die Struktur eines invasiven Projekts in den unterschiedlichen Programmiersprachen verglichen.

Einzelne C- und Rust-Programme können jeweils mithilfe von \textit{octorust} kompiliert werden.
Gleiches gilt für X10, für welches \textit{x10i} die Kompilierung durchführt.

Zusätzlich hierzu bietet \textit{octorust} jedoch auch das Kompilieren von \textit{Cargo}-Projekten.
Diese weisen zwar einen etwas
höheren Arbeitsaufwand zu Beginn des Projekts auf, erlauben aber anschließend benutzerfreundliches Einbinden von
externen Bibliotheken in der \texttt{\textsc{\textbf{Cargo.toml}}} Konfigurationsdatei. Außerdem wird in
\textit{Cargo}-Projekten die \textit{octolib}-Bibliothek automatisch eingebunden.
Im Falle von X10 und C ist eine ähnliche
Funktionalität nicht vorhanden, externe Bibliotheken müssen beim Kompilieren manuell eingebunden werden.

\subsection{Minimales Invade, Infect, Retreat}

Um den Programmieraufwand eines simplen invasiven Programms zu beurteilen, wurde ein Programm geschrieben, 
welches zuerst eine \textit{Invade}-Operation durchführt und anschließend eine simple "`Hello World"'-Funktion
auf jedem der reservierten Rechenelemente in der \textit{Infect}-Phase ausführt.
Das Programm wartet anschließend bis alle Rechenelemente mit der Ausführung beendet sind und gibt daraufhin
alle Ressourcen in der \textit{Retreat}-Phase wieder frei.

\subsubsection{Rust}

Um dieses Programm in Rust zu implementieren, muss man zunächst die notwendigen Strukturen importieren.
Benötigt werden die \texttt{\textsc{\textbf{Constraints}}}- und \texttt{\textsc{\textbf{AgentClaim}}}-Strukturen.
Zusätzlich muss der \texttt{\textsc{\textbf{libc::c\char`_void}}}-Typ importiert werden,
um die Funktionssignatur der \textit{i-let} Funktionen zu definieren.
Außerdem wird zur Ausgabe von "`Hello World!"' die \texttt{\textsc{\textbf{helper::printer::print}}}-Funktion benötigt.

Zunächst werden die \textit{Constraints} mithilfe der \texttt{\textsc{\textbf{Constraints}}}-Struktur initialisiert.
Anschließend können diese weiter konfiguriert werden oder direkt bei der Initialisierung einer Instanz der
\texttt{\textsc{\textbf{AgentClaim}}}-Struktur verwendet werden. Wird der Konstruktor
der \texttt{\textsc{\textbf{AgentClaim}}}-Struktur aufgerufen,
so wird die \textit{Invade}-Operation implizit durchgeführt.
Es muss nun nur noch die \texttt{\textsc{\textbf{infect}}}-Methode aufgerufen werden.
Dieser Methode wird eine Closure als Parameter übergeben, welche den
Text "`Hello World"' auf die Kommandozeile ausgibt. Sobald die \texttt{\textsc{\textbf{infect}}}-Methode
aufgerufen wurde, wartet der Hauptfaden des Programms darauf,
dass die einzelnen Rechenelemente ihre Ausführung beendet haben.

Hiernach muss der Programmierer keine zusätzlichen Instruktionen aufrufen. Die \textit{Retreat}-Phase wird beim
Verlassen des Geltungsbereichs automatisch durchgeführt.

Eine beispielhafte Implementierung dieses Programms wird in Listing \ref{code:rust_minimal_infect} veranschaulicht.

\lstset{basicstyle=\small}
\begin{lstlisting}[float,caption={Minimales Invade, Infect, Retreat in Rust},label=code:rust_minimal_infect]
#![no_std]

extern crate libc;
extern crate octolib;
use libc::c_void;
use octolib::helper::printer::print;
use octolib::improvements::constraints::Constraints;
use octolib::improvements::claim::AgentClaim;

#[no_mangle]
pub extern "C" fn rust_main_ilet(claim_id: u8) {

    let mut ilet_fn = |params: *mut c_void| {
        print("Hello World\n\0");
    };
    
    let mut constr = Constraints::new(4, 4);
    let mut claim = AgentClaim::new(constr);
    claim.infect(ilet_fn, None);
}
\end{lstlisting}
\lstset{basicstyle=\normalsize}

\subsubsection{X10}

Soll das Programm in X10 implementiert werden, so muss man zuerst die benötigten Klassen importieren. Es werden 
mindestens die Klassen \\
\texttt{\textsc{\textbf{invasic.constraints.PEQuantity}}},
\texttt{\textsc{\textbf{invasic.IncarnationID}}} und \\
\texttt{\textsc{\textbf{invasic.Claim}}}
benötigt. Zusätzlich muss \texttt{\textsc{\textbf{x10.io.Console}}} importiert werden
um die Ausgabe auf die Kommandozeile zu ermöglichen.

Wie bereits bei Rust werden zunächst die \textit{Constraints} spezifiziert. Hierzu
ruft man den Konstruktor der Klasse \texttt{\textsc{\textbf{PEQuantity}}} auf. Man kann mithilfe
des \texttt{\textsc{\textbf{\char`&\char`&}}}-Operators weitere \textit{Constraints}-Attribute wie beispielsweise \\
\texttt{\textsc{\textbf{TileSharing.WITH\char`_OTHER\char`_APPLICATIONS}}} mit der
\texttt{\textsc{\textbf{PEQuantity}}} verbinden.
Anschließend führt man die \textit{Invade}-Phase mithilfe der statischen
\texttt{\textsc{\textbf{Claim.invade}}}-Methode aus und erhält als Rückgabewert den erstellten \textit{Claim}.
Um darauf die \textit{Infect}-Phase zu beginnen, muss auf diesem \textit{Claim} die
\texttt{\textsc{\textbf{infect}}}-Methode verwendet werden.
Als Parameter erhält diese eine Closure, welche "`Hello World"' auf die Kommandozeile ausgibt.
Sobald die Methode aufgerufen wurde, wartet der Hauptfaden des Programms auf das Ende der Ausführung der
verwendeten Rechenelemente.

Nach erfolgreicher Ausführung der \textit{Infect}-Phase, muss dann noch manuell die 
\texttt{\textsc{\textbf{retreat}}}-Methode des \textit{Claims} aufgerufen werden.

Eine beispielhafte Implementierung dieses Programms wird in Listing \ref{code:x10_minimal_infect} veranschaulicht.

\lstset{basicstyle=\small}
\begin{lstlisting}[float,caption={Minimales Invade, Infect, Retreat in X10},label=code:x10_minimal_infect]
import x10.io.Console;

import invasic.constraints.PEQuantity;
import invasic.constraints.TileSharing;
import invasic.IncarnationID;
import invasic.Claim;

class Infect {
    public static def main(Array[String]) {
        val iletFn = (id: IncarnationID) => {
            Console.OUT.println("Hello World!");
        };

        val constraints = new PEQuantity(4, 4)
            && TileSharing.WITH_OTHER_APPLICATIONS;
        val claim = Claim.invade(constraints);
        claim.infect(iletFn);
        claim.retreat();
    }
};
\end{lstlisting}
\lstset{basicstyle=\normalsize}

\subsubsection{C}

Im Gegensatz zu Rust und X10 müssen die in C verwendeten Strukturen nicht alle explizit importiert werden.
Es muss lediglich \texttt{\textsc{\textbf{octopos.h}}} mit der Präprozessor-Anweisung
\texttt{\textsc{\textbf{include}}} verwendet werden.
Um auf die Kommandozeile ausgeben zu können, muss zusätzlich \texttt{\textsc{\textbf{stdio.h}}} importiert werden.

Die Constraints werden mithilfe der \texttt{\textsc{\textbf{agent\char`_constr\char`_create}}}-Funktion initialisiert
und müssen anschließend mit diversen \texttt{\textsc{\textbf{agent\char`_constr\char`_set\char`_...}}}-Funktionen
konfiguriert werden.
Mit dieser konfigurierten \texttt{\textsc{\textbf{constraints\char`_t}}}-Struktur kann nun eine
\texttt{\textsc{\textbf{agentclaim\char`_t}}}-Struktur mithilfe
der \texttt{\textsc{\textbf{agent\char`_claim\char`_invade}}}-Funktion initialisiert werden.

Nach dieser \textit{Invade}-Operation muss man in C manuell über alle reservierten Rechenelemente iterieren, dabei
\texttt{\textsc{\textbf{simple\char`_ilet}}}-Strukturen für diese erstellen und anschließend mit der
\texttt{\textsc{\textbf{proxy\char`_infect}}}-Funktion
die Ausführung beginnen. Als \textit{i-let}-Funktion muss eine C Funktion verwendet werden,
da die Sprache keine Closures oder ähnliche Konstrukte bietet.
Damit der Hauptfaden auf das Ausführungsende der einzelnen Rechenelemente wartet, muss
man manuell Signalstrukturen verwenden.

Nachdem die Recheneinheiten signalisiert haben, dass sie ihre Ausführung beendet haben, muss man noch manuell
die \textit{Retreat}-Phase beginnen und anschließend mit der \texttt{\textsc{\textbf{shutdown}}}-Funktion
die Ausführung des Programms beenden.

Eine beispielhafte Implementierung dieses Programms wird in Listing \ref{code:c_minimal_infect} veranschaulicht.

\lstset{basicstyle=\small}
\begin{lstlisting}[float,caption={Minimales Invade, Infect, Retreat in C},label=code:c_minimal_infect]
#include <octopos.h>
#include <stdio.h>

void signaler(void* sig) {
    simple_signal* s = (simple_signal*)(sig);
    simple_signal_signal_and_exit(s);
}

void ilet_fn(void *signal) {
    printf("Hello World!\n");
    simple_ilet answer;
    simple_ilet_init(&answer, signaler, signal);
    dispatch_claim_send_reply(&answer);
}

void main_ilet(claim_t claim_id) {

    constraints_t constr = agent_constr_create();
    agent_constr_set_quantity(constr, 4, 4, 0);
    agent_constr_set_tile_shareable(constr, 1);
    agentclaim_t claim = agent_claim_invade(0, constr);

    simple_signal sync;
    simple_signal_init(&sync, agent_claim_get_pecount(claim));

    for (int tile=0; tile < get_tile_count(); tile++) {
        int pes=agent_claim_get_pecount_tile_type(claim,  tile, 0);
        if (pes) {
            proxy_claim_t p_claim =
                agent_claim_get_proxyclaim_tile_type(
                    claim, tile, 0
                );

            simple_ilet ilets[pes];
            for (int i = 0; i < pes; ++i) {
                simple_ilet_init(&ilets[i], ilet_fn, &sync);
            }

            proxy_infect(p_claim, &ilets[0], pes);
        }
    }

    simple_signal_wait(&sync);
    agent_claim_retreat(claim);
    shutdown(0);
}
\end{lstlisting}
\lstset{basicstyle=\normalsize}

\subsubsection{Vergleich}

Die Vorgehensweisen bei Rust und X10 sind nahezu identisch. Der einzige nennenswerte Unterschied hierbei ist es, dass
man in Rust keine explizite "'Retreat"'-Operation initialisieren muss.
Das C-Programm ist im Vergleich zu den beiden anderen Programmiersprachen hingegen komplexer und somit auch
fehleranfälliger. Zudem benötigt ein solches C-Programm ungefähr doppelt so viele
Codezeilen wie die äquivalenten Rust- oder X10-Programme. Obendrein ist es in C nicht möglich, Closures oder
ähnliche Konstrukte zu verwenden, Rust und X10 unterstützen dies jedoch.
