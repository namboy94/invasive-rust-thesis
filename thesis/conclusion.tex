\chapter{Fazit und Ausblick}\label{sec:conclusion}

Im folgenden werden die Ergebnisse der Evaluation bewertet und ein Ausblick auf die Zukunft der Programmiersprache Rust
im Bezug zum invasiven Computing geboten.

\section{Fazit}

Betrachtet man die Ergebnisse der Evaluation, so erkennt man, dass Rust in gewissen Aspekten C oder X10 vorzuziehen ist.

Zum einen eliminiert Rust einige Fehlerquellen, welche in C zu undefiniertem Verhalten führen können. In Rust muss der
Programmierer nicht selbst auf solche Fehler achten und wird teils bereits vom Compiler auf Fehler hingewiesen.
Dies erlaubt es, sicherere und weniger fehleranfällige Programme zu schreiben.

Des Weiteren weist Rust meist ein besseres Laufzeitverhalten auf als X10 es tut, vor allem in Situationen
in denen der Garbage-Collector notwendig wird. Ein Garbage Collector ist zudem für Situationen, in denen Pausen nicht
erwünscht sind, beispielsweise bei Echtzeitsystemen, ein Problem, denn diese verletzt die Verfügbarkeitsbedingung des Systems.
Außerdem weist das Kompilieren mit octorust weitaus kürzere Kompilierungsdauern als der x10i-Compiler auf, welches bei
der Entwicklung von Programmen unter Umständen ein nicht insignifikantes Zeitersparnis mit sich bringen kann. Zum Gebrauch auf
Geräten mit sehr limitiertem Speicherplatz wäre Rust ebenfalls X10 vorzuziehen, da kompilierte Rust-Programme weniger Speicherplatz
in Anspruch nehmen. Allerdings führt C in dieser Hinsicht weiterhin.

Negativ zu betrachten ist die starke Abhängigkeit von der C-Schnittstelle, welche die Sicherheit von Rust teils aufgibt.
Die Funktionen der C-Schnittstelle bieten nämlich keine Garantien über die Sicherheit und die häufige Nutzung von
Void-Zeigern zum Übertragen von Daten ist ebenfalls ein Problem, welches zu Fehlern führen kann.

\section{Ausblick}

Da Rust noch eine relativ junge Programmiersprache ist, existieren noch viele Bibliotheken für diese nicht. Zwar können mithilfe
der Foreign Function Interface Bibliotheken, die in anderen Sprachen geschrieben worden sind, eingebunden werden, diese bieten jedoch
nicht die Sicherheitsgarantien wie Rust es tut. In dieser Hinsicht sollte sich die Lage jedoch in der Zukunft verbessern, wenn Rust
weiterhin gerne von Entwicklern genutzt wird und eventuell großflächiger zum Einsatz kommt.

Eine weitere Möglichkeit in der Zukunft ist die Portierung der Standarbibliothek auf die SPARC-V8 Architektur. Die Standardbibliothek
bietet einige hilfreiche Konstrukte und Funktionen, die das Programmieren erleichtern und sicherer machen. Möglich wäre es, dass die 
Entwickler der Rust Programmiersprache selbst die SPARC-V8 Architektur in der Zukunft unterstützen und so die
Standarbibliothek portiert wird, alternativ wäre es möglich dass die Standarbibliothek im Rahmen des invasiven Computing
portiert wird, sollte genug Interesse daran bestehen.

Um die starke Abhängigkeit von IRTSS' C-Schittstelle zu verringern, können weitere Abstraktionen erstellt werden, welche von den
besonderen Eigenschaften der Rust-Programmiersprache Gebrauch machen. Dass dies möglich ist, hat die Implementierung des X10-Compilers
x10i bewiesen, denn dieser spricht die exakt selbe C-Schnittstelle an, bietet jedoch robuste Abstraktionen über diese und erlaubt so
das Entwickeln von X10-Programmen ohne dass der Programmierer sich mit dieser befassen muss.
