\chapter{Grundlagen}\label{sec:basics}

Im folgenden Kapitel werden die Grundlagen der Programmiersprache Rust, der Rechnerarchitektur SPARC-V8 und des 
invasiven Computing behandelt.

\section{Rust}

Rust ist eine Programmiersprache, welche wie beispielsweise C++ direkte Kontrolle über die
unterliegende Hardware bieten soll. Dies bedeutet in der Praxis, dass diese Sprache nicht von einem
\textit{Garbage Collector} Gebrauch macht, wie es beispielsweise bei der Sprache X10 der Fall ist.
Im Gegensatz zu C++ bietet Rust jedoch
standardmäßig automatische Speichersicherheit und garantiert somit, dass Programme keine Speicherfehler enthalten.
\cite{theRustLanguage}

Rust erfreut sich seit ihrer Veröffentlichung wachsender Beliebtheit bei Programmierern aller Art. Sie wurde
bei einer Umfrage der Webseite \textit{stackoverflow.com} im Jahre 2016 als beliebteste Programmiersprache bei 
Entwicklern ermittelt\cite{stackoverflowSurvey}.

Das wohl derzeit prominenteste Projekt, welches Rust verwendet, ist der \textit{Servo Web Browser Engine}.
Dieses Projekt wurde von Mozilla Research initiiert, erhält jedoch ebenfalls Beiträge von Freiwilligen
als auch von Unternehmen wie Samsung.
Diese Software soll nach und nach im Mozilla Firefox Webbrowser integriert werden und hierbei eine bessere
Leistung und Sicherheit als vorhergehende Technologien vorweisen\cite{engineeringServo}.

\subsection{Motivation}

Eines der Kernziele der Rust Programmiersprache ist es, sichere Speicherzugriffe zu gewährleisten, ohne
für diesen Zweck einen \textit{Garbage Collector} zu verwenden. Fehlerhafte Speicherzugriffe können undefiniertes
Verhalten auslösen, welches ein Sicherheitsrisiko darstellen kann, denn diese Fehler können gezielt ausgenutzt
werden, um Schadcode auszuführen. Hierbei beziehen sich fehlerhafte Speicherzugriffe auf das Verwenden von 
Speicherregionen, die entweder ungültig sind oder bereits befreit wurden.
\cite{engineeringServo}\cite{undefinedBehaviour}

Außerdem soll Rust paralleles Rechnen unterstützen und
so möglichst effizient moderne Hardwarearchitekturen ausnutzen.
Hierfür sollen gewisse häufig auftretende Fehler in der Parallelprogrammierung gänzlich entfallen.
\cite{theRustLanguage}

Rust lehnt sich konzeptionell und syntaktisch an C-ähnliche Sprachen an, enthält jedoch auch Konzepte aus der
funktionalen Programmierung. Zusätzlich werden Sprachkonzepte und Abstraktionen geboten, die
ansonsten zumeist in höheren Programmiersprachen vorhanden sind. Hierdurch soll der Einstieg für
Programmierer, die zuvor keine oder nur wenige Erfahrungen mit Systemsprachen gesammelt haben, erleichtert werden.
\cite{engineeringServo}

Diese zusätzlichen Sicherheiten und Nutzerfreundlichkeiten sollen gleichzeitig jedoch keine Laufzeitkonsequenzen mit 
sich ziehen, denn es ist ein Ziel von Rust, 
eine ähnliche Recheneffizienz im Vergleich mit bestehenden Systemsprachen wie C oder C++ vorzuweisen.
Außerdem soll Rust es vereinfachen, mehrere Programmiersprachen miteinander zu verwenden.
Das Ziel hierbei ist es, Code aus anderen Sprachen aus einem Rust Kontext heraus zu verwenden, als auch umgekehrt.
\cite{rustBook}


\subsection{Das Typsystem}

Das signifikanteste Alleinstellungsmerkmal der Programmiersprache Rust ist ihr Typsystem. Dieses ermöglicht es
erst, die versprochenen Sicherheitsgarantien ohne zusätzliche Konstrukte wie einem \textit{Garbage Collector}
zu realisieren.\cite{rustBook}

\subsubsection{Ownership, Move-Semantik und Borrowing}

Das Typsystem integriert das Konzept der \textit{Ownership (engl. Besitz)}.
Dies ist die zentrale Besonderheit von Rusts Typsystem.
Der Grundgedanke hierhinter ist es, dass der Zugriff auf eine Speicherregion exklusiv einer Variable
zur Verfügung gestellt wird. Diese Variable wird auch als \textit{Owner (engl. Besitzer)} bezeichnet.
Zu jedem Zeitpunkt darf die Speicherregion nur einen \textit{Owner} haben.
Sobald dieser \textit{Owner} den Geltungsbereich verlässt, wird die zugehörige Speicherregion freigegeben,
die dort vorhandenen Daten können anschließend nicht mehr verwendet werden.
Dies ist der Mechanismus, der es in Rust erlaubt, Speicher ohne manuelle Befreiung oder Nutzung eines
\textit{Garbage Collectors} zu verwalten. Er stellt sicher, dass keine fehlerhaften Speicherzugriffe
durch die Nutzung von ungültigen Daten im Speicher entstehen.\cite{rustBook}
Das Konzept des \textit{Ownership} wurde von affinen Typen inspiriert.\cite{embeddedRustOS}
Solche Typen können maximal einmal verwendet werden.\cite{affineTypes}

Die \textit{Ownership} einer Speicherregion kann mithilfe der \textit{Move}-Semantik den Besitzer wechseln.
Dies kann bei Funktionsaufrufen oder auch bei Zuweisungen geschehen. Beispiele hierfür sind in Listing
\ref{code:move_semantics} zu sehen.\cite{rustBook}

\lstset{literate=%
  {Ö}{{\"O}}1
  {Ä}{{\"A}}1
  {Ü}{{\"U}}1
  {ß}{{\ss}}1
  {ü}{{\"u}}1
  {ä}{{\"a}}1
  {ö}{{\"o}}1
}
\begin{lstlisting}[float,caption={Beispieldarstellung der \textit{Move}-Semantik},label=code:move_semantics]
let x = String::from("hello");
// Owner: x

let y = x;
// Owner: y
// x ist nun ungültig

let z = f(y) 
// Owner: -
// Ownership wurde an f übergeben,
// da f aber bereits den Geltungsbereich verlassen hat
// ist der Speicherbereich nun befreit.
// y ist nun ungültig
\end{lstlisting}

Zusätzlich zur \textit{Move}-Semantik können in Rust Referenzen zu Variablen verwendet werden. Dies wird im Kontext
von Rust häufig als \textit{borrowing (engl. ausleihen)} bezeichnet. Verwendet man eine Referenz auf eine Variable,
ist der \textit{Owner} des Speicherbereichs weiterhin die originale Variable.
Zur Compile-Zeit wird überprüft, dass alle Referenzen gültig sind, um so die Existenz von hängenden Zeigern
zu verhindern. Eine Referenz kann also nie auf eine Variable zeigen, die bereits den Geltungsbereich verlassen hat.
\cite{rustBook}

Es gibt zwei unterschiedliche
Typen von Referenzen, veränderliche und unveränderliche Referenzen. Standardmäßig werden in Rust unveränderliche
Referenzen verwendet. Diese erlauben es nicht, den Wert des Speicherbereichs zu verändern, sondern nur auszulesen.
Es gibt keine Limitierung bei der Anzahl unveränderlicher Referenzen die eine Variable besitzen kann. Im Gegensatz
hierzu kann nur eine veränderliche Referenz auf die Variable existieren. Dies ermöglicht es bereits zur
Compile-Zeit gewisse Wettlaufsituationen, sogenannte \textit{Data Races},
in der Parallelprogrammierung auszuschließen.\cite{rustBook}

\subsubsection{Strukturen, Implementierungen, Aufzählungen und Traits}

Zum Erstellen von benutzerdefinierten Typen bietet Rust Strukturen (\texttt{\textsc{\textbf{struct}}}),
Implementierungen (\texttt{\textsc{\textbf{impl}}}), Aufzählungen (\texttt{\textsc{\textbf{enum}}}) und
Traits (\texttt{\textsc{\textbf{trait}}}).

Strukturen in Rust spezifizieren benutzerdefinierte Datentypen, welche verwandte Daten vereinen. Das Konzept
ähnelt den Strukturen in C. Anders wie in C kann man jedoch mithilfe von Implementierungen Methoden und assoziierte
Funktionen für diese Strukturen definieren.
Diese Möglichkeit erlaubt es eine objektorientierte Herangehensweise in Rust zu verwenden.
Implementierungen erlauben das Definieren von Methoden, welche auf einer initiierten Struktur aufgerufen werden 
können und eine Referenz auf \texttt{\textsc{\textbf{self}}}, also der Struktur selbst, als Parameter entgegennehmen.
Zusätzlich können Funktionen definiert werden, welche keinen solchen Parameter entgegennehmen und somit auch mit
Strukturen verwendet werden können ohne diese im Voraus zu initialisieren.
Dies ähnelt statischen Methoden in objektorientierten Sprachen.\cite{rustBook}

Aufzählungen agieren in Rust ähnlich zu anderen Programmiersprachen. Der Vorteil der Aufzählungen in Rust ist es,
dass diese fürs \textit{Pattern Matching} verwendet werden können.
Ein Beispiel dazu, wie dies in der Praxis verwendet
werden kann, wird in Kapitel \ref{sec:error_handling} beschrieben.
\cite{rustBook}

Traits sind Sprachkonstrukte in Rust, welche das Verhalten von Typen abstrakt definieren können. Sie ähneln 
hierbei Schnittstellen (\textit{Interfaces}) aus anderen Programmiersprachen.
Zusätzlich zur benutzerdefinierten Spezifikation
solcher \textit{Interface}-ähnlichen Konstrukte bietet Rust vordefinierte Traits, welche das Verhalten der Typen
bezüglich \textit{Ownership} und \textit{Move}-Semantik beeinflussen können.\cite{rustBook}

Ein solches Trait ist das \textit{Drop}-Trait. Dieses erlaubt es, das Verhalten eines Typen beim Verlassen des
Geltungsbereichs zu beeinflussen. In der \textit{drop}-Methode des Traits wird es ermöglicht, einen Destruktor
für den Typ zu definieren. Dieser kann aus unterschiedlichen Gründen benötigt oder hilfreich sein.
Beispielsweise kann einen solcher Destruktor verwendet werden, 
um einen Referenzzähler zu dekrementieren, unbenutzte Netzwerkverbindungen
zu trennen oder andersweitig Ressourcen wieder freizugeben.\cite{rustBook}

Außerdem gibt es die \textit{Copy}-Traits, welche es ermöglichen, anstelle der \textit{Move}-Semantik die
\textit{Copy}-Semantik zu verwenden. Für Typen, die dieses Trait implementieren, wird jedes mal, wenn ein
Speicherbereich im Kontext der \textit{Move}-Semantik den \textit{Owner} wechseln würde,
stattdessen eine bitweise Kopie der Daten erstellt. Primitive Datentypen wie \texttt{\textsc{\textbf{i32}}}
verhalten sich standardmäßig bereits so. Das \textit{Copy}-Trait kann jedoch nicht auf jeden Typ
angewandt werden. Insbesondere kann das \textit{Drop}-Trait nicht gleichzeitig auf diesen Typ angewandt werden,
da diese beim Verlassen des Geltungsbereichs einen Einfluss außerhalb ihres Speicherbereich haben können.
Würde der Destruktor für jede Kopie aufgerufen werden, könnte dies zu unvorhergesehenen und
unerwünschten Konsequenzen führen.\cite{rustBook}\cite{rustDocCopy}

\subsection{Fehlerbehandlung}\label{sec:error_handling}

Obwohl der Compiler dank Rusts Typsystem bereits bei der Kompilierung eine Vielzahl von Fehlern erkennen kann,
ist dies  nicht für alle möglichen Fehlertypen möglich. Beispielsweise ist es möglich, dass
durch eine inkorrekte Indizierung eines Arrays ein Fehler entsteht.
Solche Fehler müssen zur Laufzeit behandelt werden.\cite{rustBook}

Fehler werden in Rust idiomatisch mithilfe von Rückgabewerten vom Aufzählungstyp \texttt{\textsc{\textbf{Result}}}
behandelt. Anhand dieser Rückgabewerte lässt sich mithilfe von
\textit{Pattern Matching} prüfen, ob ein Fehler beim Funktionsaufruf aufgetreten ist. Ist keiner aufgetreten,
kann der eigentliche Rückgabewert der Funktion aus dieser Aufzählung entnommen werden.
\cite{rustBook}

Wird ein Fehler nicht auf diese Art und Weise behandelt, wird das \texttt{\textsc{\textbf{panic!}}}-Makro
aufgerufen. Dieses befreit standardmäßig die Daten des Programms auf dem Stack, druckt anschließend einen
Stacktrace und beendet darauf die Ausführung des Programms.
Es ist jedoch auch möglich, ein Programm so zu konfigurieren,
dass stattdessen die Ausführung des Programms beendet wird sobald ein unbehandelter Fehler auftritt.\cite{rustBook}

\subsection{Interaktion mit anderen Programmiersprachen}

Eine praktische Eigenschaft Rusts ist es, dass sie in Verbindung mit einer Vielzahl von anderen Programmiersprachen
verwendet werden kann. Mithilfe der \textit{Foreign Function Interface} kann man Code aus anderen
Sprachen in Rust verwenden. Dadurch kann Rust beispielsweise Bibliotheken, welche in C geschrieben sind, verwenden.
Die \textit{Foreign Function Interface} erlaubt es zusätzlich, Rust Code aus dem Kontext anderer Sprachen zu
verwenden. Dies wird für eine Vielzahl an Programmiersprachen unterstützt, unter anderem C, aber auch höheren
Programmiersprachen wie Python oder Ruby, welche durch die Einbindung von Rust-Code Leistungssteigerungen erzielen können.\cite{firstRustBook}\cite{rustBook-1.2.0}

\subsection{Unsicheres Rust}

In gewissen Situationen kann es sein, dass Rusts Sicherheitsgarantien die Implementierung erschweren oder
sogar unmöglich machen. Vor allem bei sehr Hardware-nahen Programmen, beispielsweise einem Betriebssystem,
kann eine solche Situation vorkommen. Als Lösung für dieses Problem bietet Rust
\texttt{\textsc{\textbf{unsafe}}}-Blöcke an. Innerhalb dieser ist es möglich, Zeiger zu dereferenzieren, welche
direkt ein Speichersegment adressieren. Somit hat man direkte Kontrolle über diesen Speicher, was
jedoch bei Unachtsamkeit seitens des Entwicklers zu Fehlern oder undefiniertem Verhalten führen kann.
Außerdem können unsichere Funktionen, beispielsweise welche die mithilfe der
\textit{Foreign Function Interface} aus anderen Programmiersprachen importiert wurden,
nur in \texttt{\textsc{\textbf{unsafe}}}-Blöcken verwendet werden.\cite{rustBook}

Auch in \texttt{\textsc{\textbf{unsafe}}}-Blöcken bleibt das Konzept von \textit{Ownership} und die zugehörige
\textit{Move}-Semantik und \textit{Borrowing} bestehen. Diese Sicherheitsmechanik wird also nicht aufgegeben
wenn man die Funktionalität der \texttt{\textsc{\textbf{unsafe}}}-Blöcke verwendet.\cite{rustBook}

\subsection{Architektur/Compiler}

Der offizielle Rust Compiler heißt \texttt{\textsc{\textbf{rustc}}} und kann eigenständige Rust-Quelldateien
kompilieren. Rust-Quelldateien haben per Konvention die Dateiendung \texttt{\textsc{\textbf{.rs}}}.\cite{rustBook}

Als Lösung zum Abhängigkeitsmanagement und der Distribution wurde das Werkzeug \texttt{\textsc{\textbf{cargo}}} 
entwickelt. Dieses ermöglicht es dem Programmierer, verwendete Bibliotheken in ein Projekt einzugliedern,
indem diese in einer \texttt{\textsc{\textbf{Cargo.toml}}}-Konfigurationsdatei im Hauptverzeichnis
des Projekts angegeben werden. Hierbei ist es möglich, Bibliotheken von der zentralen \textit{crates.io}
Plattform automatisch herunterladen zu lassen oder einen lokalen Dateipfad zu einer solchen
Bibliothek anzugeben.
Außerdem unterstützt \texttt{\textsc{\textbf{cargo}}} hilfreiche Funktionen zum Testen oder
Verteilen der entwickelten Software. Kompilierte Software kann als ein sogenanntes \textit{Crate} bei
\textit{crates.io} hochgeladen werden.\cite{rustBook}

Um den Gebrauch von unterschiedlichen \texttt{\textsc{\textbf{rustc}}} und \texttt{\textsc{\textbf{cargo}}} Versionen 
zu vereinfachen, wurde das Werkzeug \texttt{\textsc{\textbf{rustup}}} entwickelt. Dieses
Hilfsprogramm ermöglicht es, unterschiedliche Varianten des Rust-Compilers zu installieren und bei Belieben zu
wechseln. Es bietet auch die Funktionalität, die benötigten Kern- und Standardbibliotheken für unterstützte
Architekturen herunterzuladen, um Programme für andere Zielarchitekturen zu kompilieren.\cite{rustupRepo}

Sowohl \texttt{\textsc{\textbf{rustc}}} als auch \texttt{\textsc{\textbf{cargo}}} ermöglichen das Kompilieren für
unterschiedliche Zielarchitekturen mithilfe der
\texttt{\textsc{\textbf{-{}-target}}}-Option. Als Compiler-Backend wird \textit{LLVM} verwendet.
Rust-Code wird also vor der Übersetzung zu Maschinenbefehlen in die Zwischensprache
\textit{LLVM-IR} übersetzt und erst anschließend weiterverarbeitet.
Somit ist es möglich, Rust-Programme für alle von \textit{LLVM}
unterstützten Rechnerarchitekturen zu kompilieren.\cite{rustGPU}




%\section{Sicherheit}

%Im folgenden wird die Sicherheit bezüglich undefiniertem Verhalten als Folge von fehlerhaften Speicherzugriffen analysiert.
%Hierbei ist vor allem der Vergleich zwischen Rust und C interessant, da einige von Rusts primären Eigenschaften die
%Risiken der C-Programmierung beseitigen sollen.

%\subsection{Division durch 0}

%Dividiert man in C einen Wert durch die Zahl 0, kann es zu undefiniertem Verhalten führen, welches auf jeden Fall unerwünscht
%ist. Versucht man dies in Rust, so bricht das Programm sofort ab, indem die "`panic\char`_fmt"'-Funktion aufgerufen wird und es
%kann nicht zu undefiniertem Verhalten kommen.

%\subsection{Pufferüberlauf}

%Ein weiterer Fall, der in C zu undefiniertem Verhalten führt, sind Pufferüberläufe. Diese geschehen, wenn man beispielsweise
%in einem Array der Größe n versucht, auf das n+1te Element zuzugreifen. Wie bereits im letzten Vergleich kann dies in 
%Rust nicht geschehen, da das Programm mit einem Aufruf der "`panic\char`_fmt"'-Funktion die Ausführung beendet.

%\subsection{Nicht Initialisierte Variablen}

%In C ist es möglich, uninitialiserte Variablen zu verwenden, dies führt allerdings ebenfalls zu undefiniertem Verhalten.
%In Rust hingegen wird dies bereits vom Compiler verhindert, da er den Gebrauch von uninitialisierten Variablen verbietet.
%Ein Rust-Programm welches also uninitialisierte Variablen verwendet kompiliert also gar nicht und kann so natürlich auch nicht
%zu undefiniertem Verhalten führen.




\section{SPARC und SPARC-V8}

SPARC (\textbf{S}calable \textbf{P}rocessor \textbf{ARC}hitecture) ist eine Rechnerarchitektur, welche auf das
RISC (Reduced Instruction Set Computing) Konzept aufbaut. Eines der Ziele der SPARC Architektur ist es, skalierbar
zu sein und so in unterschiedlichsten Umgebungen zum Einsatz zu kommen. So können SPARC Prozessoren in
Mikrocontrollern bis hin zu Supercomputern verwendet werden. Die Architektur wird an unterschiedliche
Halbleiter-Unternehmen lizensiert, mit dem Ziel durch konkurrierende Kräfte Innovation bezüglich Leistung und
Kosten zu fördern.\cite{sparc}

SPARC-V8 ist eine auf SPARC basierende Architektur. Es handelt sich hierbei um eine Allzweckarchitektur mit einem
32-bit breiten Datenpfad.\cite{sparcv8Eval}

\subsection{LEON}

Die ursprünglich von der European Space Agency (ESA) und anschließend von Gaisler Research entwickelten
LEON-Prozessoren basieren auf der SPARC-V8 Architektur.
Das Hauptziel der LEON-Prozessoren war es, fehlertolerante Nachfolger zu den älteren SPARC-V7-basierten
Prozessoren, die zu dem Zeitpunkt bei der ESA Verwendung fanden, zu bieten.\cite{gaislerLeon}
Es handelt sich bei der LEON-Prozessorfamilie um stark konfigurierbare
Implementierungen der SPARC-V8 Architektur in der Hardwarebeschreibungssprache
VHDL (Very High Speed Integrated Circuit Hardware Description Language).
Unterschiedliche Iterationen der Prozessorfamilie sind unter Open-Source Lizenzen als VHDL-Designs verfügbar.
So wurden die VHDL-Designs des LEON2 Prozessors unter der GNU LGPL Lizenz veröffentlicht\cite{sparcv8Eval} und
auch das Design des LEON3 Prozessors ist unter einer Open-Source Lizenz verfügbar\cite{invasiveArrays}.

Die Kombination von stark konfigurierbaren Designs und der Open-Source Lizenz ermöglicht den Gebrauch der
LEON VHDL Designs in domänenspezifischen, angepassten ASICs (Application-Specific Integrated Circuit),
FPGAs (Field Programmable Gate Array) oder SOCs (System On Chip). Daher eignet sich die LEON Prozessorfamilie
auch zum Einsatz im invasiven Computing, um deren neuartigen Konzepte möglichst effizient auszunutzen.
\cite{sparcv8Eval}

\section{Invasives Computing}

Invasives Computing beschreibt die Möglichkeit eines Programms auf einem Parallelrechner,
dynamisch Hardwareressourcen zu reservieren, nutzen und anschließend wieder freizugeben.\cite{octopos}

Die drei fundamentalen Phasen des invasiven Computing sind die \textit{\textbf{Invade}}, \textit{\textbf{Infect}}
und \textit{\textbf{Retreat}} Phasen. In der \textit{Invade} Phase werden zunächst die gewünschten
Hardwareressourcen angefordert und anschließend reserviert. Diese Menge an Ressourcen kann dann mithilfe
eines \textit{Claims} verwaltet werden. Die zu reservierenden Ressourcen werden mithilfe von \textit{Constraints}
spezifiziert.
Wurden die Ressourcen erfolgreich reserviert, kann die \textit{Infect}
Phase beginnen. In dieser werden Ausführungsfäden erstellt,
welche auf den reservierten Hardwareressourcen ausgeführt werden. Im Kontext des invasiven Computing wird ein solcher
Faden als \textit{invasive-let} oder abgekürzt als \textit{i-let} bezeichnet\cite{invasiveCommonTerms}.
Benötigt das Programm die reservierten Ressourcen nicht mehr, können diese in der \textit{Retreat} Phase wieder
freigegeben werden.\cite{octopos}

Für den praktischen Einsatz des invasiven Computings wurde ein Betriebssystem namens \textit{OctoPOS} entwickelt. 
Dieses unterstützt die \textit{Invade}, \textit{Infect} und \textit{Retreat} Phasen auf der Systemebene.
Zusätzlich hierzu existiert das \textit{invasive Run-Time Support System (iRTSS)}. Diese Middleware
fungiert als eine Hardware-Abstraktionsschicht und als Ressourcenverwalter.
\textit{OctoPOS} und \textit{iRTSS} bieten so eine Plattform für das invasive Programmieren und
unterstützen derzeit den Gebrauch der Programmiersprachen C, C++ und X10.
\cite{octopos}\cite{invasiveManyCore}\cite{invasiveRISC}

Um das Konzept des invasiven Computing möglichst effizient auszunutzen, wird speziell an das invasive Computing
angepasste Hardware entwickelt, deren Prozessorkerne auf dem Design des LEON3 Prozessors basieren.
Diese Hardware unterstützt gewisse invasive Grundfunktionen und entlastet somit das \textit{OctoPOS} Betriebssystem
und das \textit{iRTSS}.\cite{invasiveArrays}

\section{Verwandte Arbeiten}

Eine verwandte Arbeit ist "`Invasive Computing—An Overview"'\cite{invasiveOverview}
von Jürgen Teich, Jörg Henkel, Andreas Herkersdorf, Doris Schmitt-Landsiedel, Wolfgang Schröder-Preikschat
und Gregor Snelting. Diese illustriert die Grundkonzepte des invasiven Computing.

Eine weitere verwandte Arbeit ist "`Technical Report: An X10 Compiler for Invasive Architectures"'\cite{invasiveX10} 
von Sebastian Buchwald, Manuel Mohr und Andreas Zwinkau, welche die Erstellung eines X10 Compilers für den Gebrauch 
im invasiven Computing beschreibt.
