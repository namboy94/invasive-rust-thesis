\chapter{Grundlagen und Verwandte Arbeiten}\label{sec:basics}

Im folgenden werden die Grundlagen der Programmiersprache Rust, des invasiven Computing und der Rechnerarchitektur
SPARC behandelt.

\section{Rust}


Rust ist eine Programmiersprache, welche von Mozilla Research entwickelt wird \cite(rustWikDe).
Die Entwicklung der Sprache begann als persönliches Projekt des Mozilla-Mitarbeiters Graydon Hoare und wird
seit 2009 von Mozilla unterstützt \cite(rustWikDe). 2010 erschien die erste öffentliche Version der Sprache,
im Jahr 2015 wurde die erste stabile Version veröffentlicht \cite(rustWikDe). 
Sie erfreut sich in der jüngsten Vergangenheit wachsende Beliebtheit bei Programmierern aller Art. So wurde
bei einer Umfrage der Webseite stackoverflow.com im Jahre 2016 als beliebteste Programmiersprache bei Entwicklern
ermittelt. \cite{stackoverflowSurvey}

Das wohl derzeit prominenteste Projekt, welche Rust verwendet ist der Servo Layout-Engine, welcher gemeinsam von
Mozilla und Samsung\cite{servoWikiDe} entwickelt wird. Dieser Layout-Engine soll schrittweise
den bisherigen Engine `Gecko`, welcher in C++ geschrieben wurde, im Mozilla Firefox Webbrowser ersetzen und dabei
deutlich bessere Leistung vorweisen.

\subsection{Motivation}

Eine der Kernziele der Rust Programmiersprache ist es, sichere Speicherzugriffe zu gewährleisten. Dies wird mithilfe des sogenannten
``Ownership''-modells erreicht. Dieses Modell ermöglicht es, fehlerhafte Speicherzugriffe zu vermeiden, welche ein ernsthaftes
Sicherheitsrisiko darstellen, denn diese können zu undefiniertem Verhalten führen, welches wiederum auch gezielt als
Angriffsvektor genutzt werden kann. Diese Speichersicherheit wird erreicht, ohne dabei von einem Garbage Collector Gebrauch
zu machen. Dies macht Rust zu einer interessanten Alternative zu Systemsprachen
wie C, welche nicht sicher bezüglich der Speicherverwaltung sind, als auch zu Sprachen mit Garbage Collector wie X10, welche durch
den Garbage Collector in gewissen Situationen und Anwendungsbereichen, beispielsweise bei Echtzeitsystemen, nicht ideal sind.

Eine weitere Angriffsfläche als auch Fehlerquelle in herkömmlichen Systemprogrammiersprachen wie C und C++
ist das unsichere Typsystem. In Rust wird dies vermieden, indem ein typsicheres Typsystem verwendet wird.

Ein weiteres Ziel der Sprache ist die Geschwindigkeit. Rust soll vergleichbare Geschwindigkeiten zu anderen Systemsprachen wie
C oder C++ erreichen. Dies kann in Benchmarks zumeist auch bestätigt werden\cite{rustVsCBenchmark}. Benchmarks, welche Rust mit
höheren Programmiersprachen wie Java\cite{rustVsJavaBenchmark} oder Python\cite{rustVsPythonBenchmark} vergleichen,
weisen oft einen signfikanten Vorspruch für Rust auf.

Das effiziente Einbinden mit anderen Sprachen ist eine zusätzliche Motivation hinter der Programmiersprache.
So kann Rust beispielsweise mithilfe der sogenannten ``Foreign Function Interface''
Funktionen welche in anderen Programmiersprachen geschrieben sind benutzen. Außerdem ist es möglich,
Rust Code aus anderen Programmiersprachen, inklusive höheren Programmiersprachen wie Python oder Ruby, aufzurufen.

Nebenläufigkeit ist ein weiteres Ziel der Sprache, mithilfe des Architektur der Sprache soll es zwischen einzelnen Threads
nicht zu kritischen Wettläufen kommen, welches eines der Hauptfehlerquellen bei parallelem Programmieren beseitigt.

Rust soll dem Programmierer zudem höhere Abstraktionen bieten, welche nicht oder nur gering die Effizienz des Programm
beinflussen. Diese sogenannten ``Zero-cost Abstractions'' erleichtern es Programmierern, welche nur wenig Erfahrung mit
Systemsprachen vorweisen können, die Programmiersprache zu verwenden, ohne dabei auf Leistung verzichten zu müssen..

\subsection{Grundlegende Eigenschaften der Sprache}

Im folgenden werden die grundlegenden Eigenschaften der Programmiersprache Rust erläutert.

\subsubsection{Das Ownership-Modell}

Das signifikanteste Alleinstellungsmerkmal der Programmiersprache Rust ist das Ownership Modell. Erst hierdurch wird die
Speichersicherheit ohne Garbage Collector möglich.

% Gültigen Anwendungsbereich verlassen ~ Out of Scope?
Die Grundlage des Ownership-Modells ist die Bindung von Speicher an eine Variable. Sobald eine Variable initialisiert wird, 
wird Speicher für diese alloziert. Sobald diese Variable sich allerdings nicht mehr im gültigen Anwendungsbereich befindet,
wird dieser Speicher automatisch wieder freigegeben. Dies wird im folgenden Beispiel veranschaulicht:

\begin{verbatim}
    fn foo() {
        let v = vec![1, 2, 3]; // Alloziert Speicher
    }

    fn main() {
        foo();
        // Speicher ist hier wieder freigegeben
    } 
\end{verbatim}

Außerdem erlaubt es Rust keinen zwei Variablen auf denselben Speicher zu zeigen. So wird beispielsweise bei einer Zuweisung einer
Variable zu einer anderen der ``Besitz'' des Speichers auf die neue Variable übertragen und die alte Variable kann nun nicht mehr
verwendet werden. Dies geschieht auch wenn die Variable als Parameter in einem Funktionsaufrauf verwendet wird.
Dieses Verhalten ist als ``Move''-Semantik bekannt. Eine Veranschaulichung dessen folgt:

\begin{verbatim}
    fn foo(v: Vec<i32>) {
        // Details unwichtig
    }

    fn main() {
	
        let v = vec![1, 2, 3];
        let v2 = v;  // v ist ab jetzt nicht mehr benutzbar!
        foo(v2);     // v2 ist ab jetzt nicht mehr benutzbar!

    }
\end{verbatim}

Durch dieses Verhalten wird sichergestellt, dass eine Variable exklusiven Zugriff auf die allozierten Speicherbereiche besitzt,
wodurch fehlerhafte Speicherzugriffe vermieden werden.

Eine Ausnahme hierzu bilden Type, welche das ``Copy-Trait'' implementieren. So können beispielsweise variablen vom Typ i32, welches
das ``Copy-Trait'' implementiert, beliebig oft wiederverwendet werden. In diesem Fall existieren trotzdem nicht mehrere Zeiger
auf dieselbe Speicherregion, der Speicherinhalt wird stattdessen jedesmal kopiert.

Ein weiteres interessantes ``Trait'' welches ein Typ in Rust implementieren kann ist das ``Drop-Trait''. Imlementiert ein Typ dieses,
so kann zusätzlicher Code ausgeführt werden, sobald eine Variable von dem Typ den gültigen Anwendungsbereich verlässt. Dies kann
hilfreich sein, um andere Speicherbereiche, welche in einem nicht-trivialen Zusammenhang von der nun ungültigen Variable abhängen.

Das Ownership-Modell wird in der Praxis von Compiler durchgesetzt. So werden die Garantien, welche das Ownership-Modell
verspricht, bereits zur Compile-Zeit sichergestellt. Zusätzlich fällt die Notwendigkeit für einen Garbage Collector weg, welches
zu einer besseren Laufzeitseffizienz und Determinismus beitragen.

\subsubsection{Typsystem}

Rust verfügt über ein statisches, typsicheres Typsystem. Dies bedeutet dass jede Variable einen eindeutigen Typ besitzt und keinen
anderen Typ annehmen kann.

Rust erlaubt es zudem, per Typinferenz die explizite Angabe eines Typen bei der Variablendeklaration zu vermeiden.

Typsystem - Hindley-Milner

Typsicher

Lineare Typen / -> Affine Typen <-

Generics

\subsubsection{Paradigmen}



Mehrere Pardigmen (OOP, Funktional)

Die Programmiersprache Rust lehnt sich syntaktisch an andere C-artige Sprachen an, unterscheidet sich aber hierbei in einigen
Aspekten. So sind in `if`/`else` Blöcken beispielsweise die runden Klammern beispielsweise Optional, so wie es auch in
Sprachen wie Python der Fall ist.

\subsubsection{Fehlerbehandlung}


Keine Exceptions -> Result<T,U>



Keine Race Conditions - Sync/Send

\subsection{Architektur/Compiler}

Der offizielle Rust Compiler rustc 

Als Lösung zum Abhängigkeitsmanagement und der Distribution wurde das Tool `cargo` entwickelt. Es ermöglicht es
dem Programmierer verwendete Bibliotheken in ein Projekt einzugliedern, indem diese in einer `.toml` Datei im Hauptverzeichniss
des Projekts angegeben werden. Außerdem unterstützt `cargo` mehrere weiter hilfreiche Funktionen zum testen oder
distributieren der entwickelten Software. Kompilierte Software kann als ein sogenanntes `Crate` bei der
von den Rust entwickelten Platform[Citation Needed] `crates.io` mithilfe von `cargo` hochgeladen werden.

Etwas zum Cross Compiling

LLVM



\section{SPARC}

Sparc

\section{Invasives Computing}

Invasives Computing ist ein paralleles Programmiermodell, welches es ermöglicht, temporär Ressourcen auf einem Parallelrechner
zu beanspruchen und anschließend wieder freizugeben.Hierbei gibt es drei wesentliche Phasen, die `invade`, `infect` und `retreat` Phasen. 

In der `invade` Phase werden zunächst Ressourcen für das laufende Programm reserviert. Welche Ressourcen genau reserviert werden,
wird durch `constraints` bestimmt.

Anschließend wird in der `infect` Phase die Funktion auf den reservierten Prozessorelementen ausgeführt.

Werden die reservierten Ressourcen nicht mehr benötigt, so kann man diese in der `retreat` Phase wieder freigeben.

Von WEM? https://invasic.informatik.uni-erlangen.de/en/index.php
