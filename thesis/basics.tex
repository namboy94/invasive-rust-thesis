\chapter{Grundlagen und Verwandte Arbeiten}\label{sec:basics}

\section{Rust}

Rust ist eine Programmiersprache, welche von Mozilla Research entwickelt wird \cite(rustWikDe).
Die Entwicklung der Sprache begann als persönliches Projekt des Mozilla-Mitarbeiters Graydon Hoare und wird
seit 2009 von Mozilla unterstützt \cite(rustWikDe). 2010 erschien die erste öffentliche Version der Sprache,
im Jahr 2015 wurde die erste stabile Version veröffentlicht \cite(rustWikDe). 
Sie erfreut sich in der jüngsten Vergangenheit wachsende Beliebtheit bei Programmierern aller Art.
%TODO Statistik dass Rust beliebt ist


\subsection{Motivation}

Eine der Kernziele der Rust Programmiersprache ist es, sichere Speicherzugriffe zu gewährleisten. Dies wird mithilfe des sogenannten
``Ownership''-modells erreicht. Dieses Modell ermöglicht es, fehlerhafte Speicherzugriffe und Pufferüberläufe zu vermeiden, ohne
dabei von einem Garbage Collector Gebrauch zu machen. Dies macht Rust zu einer interessanten Alternative zu Low-Level Sprachen
wie C, welche nicht sicher bezüglich der Speicherverwaltung sind, als auch zu Sprachen mit Garbage Collector wie X10, welche durch
den Garbage Collector in gewissen Situationen, beispielsweise bei Echtzeitsystemen, nicht ideal sind.

Ein weiteres Ziel der Sprache ist die Geschwindigkeit. Rust soll vergleichbare Geschwindigkeiten zu anderen Low-Level Sprachen wie
C erreichen und obendrein effizientes Einbinden mit anderen Sprachen ermöglichen.

Nebenläuigkeit ist ein weiteres Hauptziel der Sprache.

Rust bietet Abstraktionen, welche im Gegensatz zu Sprachen wie Python, nicht oder nur gering die Geschwindigkeit des Programm beinflussen.

\subsection{Grundlegende Eigenschaften der Sprache}

Die Programmiersprache Rust lehnt sich syntaktisch an andere C-artige Sprachen an, unterscheidet sich aber hierbei in einigen
Aspekten. So sind in `if`/`else` Blöcken beispielsweise die runden Klammern beispielsweise Optional, so wie es auch in
Sprachen wie Python der Fall ist.

Typsystem

Mehrere Pardigmen (OOP, Funktional)

\subsection{Eigentums/Ownership Modell}

\subsection{Architektur/Compiler}

Der offizielle Rust Compiler rustc

Zum Abhängigkeitsmanagement und Distribution wurde das Tool `cargo` entwickelt. Es ermóglicht es 

Etwas zum Cross Compiling



\section{SPARC}

Sparc

\section{Invasives Computing}

Invasives Computing ist ein paralleles Programmiermodell, welches es ermöglicht, temporär Ressourcen auf einem Parallelrechner
zu beanspruchen und anschließend wieder freizugeben.Hierbei gibt es drei wesentliche Phasen, die `invade`, `infect` und `retreat` Phasen. 

In der `invade` Phase werden zunächst Ressourcen für das laufende Programm reserviert. Welche Ressourcen genau reserviert werden,
wird durch `constraints` bestimmt.

Anschließend wird in der `infect` Phase die Funktion auf den reservierten Prozessorelementen ausgeführt.

Werden die reservierten Ressourcen nicht mehr benötigt, so kann man diese in der `retreat` Phase wieder freigeben.
